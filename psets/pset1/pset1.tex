%%%%%%%%%%%%%%%%%%%%%%%%%%%%%% Preamble
\documentclass{article}
\usepackage{amsmath,amssymb,amsthm,fullpage}
\usepackage[a4paper,bindingoffset=0in,left=1in,right=1in,top=1in,
bottom=1in,footskip=0in]{geometry}
\newtheorem*{prop}{Proposition}
%\newcounter{Examplecount}
%\setcounter{Examplecount}{0}
\newenvironment{discussion}{\noindent Discussion.}{}
\setlength{\headheight}{12pt}
\setlength{\headsep}{10pt}
\usepackage{fancyhdr}
\pagestyle{fancy}
\fancyhf{}
\lhead{Ma6a Pset 1}
\rhead{Matt Lim}
\pagenumbering{gobble}
\begin{document}
%%%%%%%%%%%%%%%%%%%%%%%%%%%%%% Problem 1
\section*{Problem 1}
We will first prove the prime decomposition property by contradiction.
Assume to the contrary that "every natural number $n \geq 2$ is either a prime or
a product of primes" is not true. So, consider the set of natural numbers $S$
that are not primes or product of primes. By the well ordering principle,
there is a least number in this set. Call this number $n$. We have that $n$
is not a product of primes. This means that one of its factors has to be
non-prime and not be a product of primes, which means it is in $S$. But this
is a contradiction since $n$ is the least number in $S$, so it cannot have
a factor that is greater than it.

We will now use induction to prove that there are infinitely many primes.
To do this, we will prove that given a set of primes of size $n-1$, we can generate
a set of primes of size $n$.
Our base case will be a set of primes of size $1$. We will call this set
$S$. If we multiply all the numbers
in this set together and add $1$, we can see that this number is either a prime
that is not in $S$, or is a product of a prime that is not in $S$.
This is because of prime decomposition, and because the number is larger
than all the primes in $S$ and not a factor of any of the primes in $S$. Either way,
we can add this new prime to $S$ to make a set of primes of size $2$. So the base
case holds. Now, for the inductive assumption. Assume that given a set of
primes of size $k-1$, we can generate a set of primes of size $k$,
for all $1 \leq k \leq n$. Given this assumption, we have that there are sets of primes
of size $n$. Now we must show that given a set of primes of size $n$, we can
generate a set of primes of size $n+1$. So, let us consider an arbitrary set of
primes $S$ of size $n$. To generate a prime number not in $S$, multiply all
elements of $S$ together and add $1$. We can see that this number is either a
prime that is not in $S$, or is a product of a prime that is not in $S$. This
is true because of prime decomposition, and because the number is larger than
all the primes in $S$ and not a factor of any of the primes in $S$. Either way,
we can add this new prime to $S$ to make a set of primes of size $n+1$.
Thus, by induction, we can conclude that given a set of primes of size $n-1$, we
can generate a set of primes of size $n$. By this inductive proof, it is clear
that there are infinitely many primes. To see why, assume to the contrary that there
are not, and that there exist only some finite set of primes of size $n$. But then,
given what we proved, we can generate a set of primes of size $n+1$, which is
a contradiction.
%%%%%%%%%%%%%%%%%%%%%%%%%%%%%% Problem 2
\section*{Problem 2}
Assume to the contrary that the pair $q,r$ is not unique. Then we have that
some other pair $q_2, r_2 \in \mathbb{N}$ exists such that $a = bq_2 + r_2$
and $0 \leq r_2 \leq b$. Consider the case where $q_2 > q$; that is, $q_2 = q + x$,
where $x \in \mathbb{N}$ and $x \geq 1$. We also have that $r_2 = r + y$,
where $y \in \mathbb{Z}$. Then we have that
\[ a = bq_2 + r_2 \]
\[ a = b(q + x) + (r + y) \]
\[ a = bq + bx + (r + y) \]
We have that $y = -bx \leq -b$, which means that $(r + y) = r_2 < 0$, since $r < b$.
But this is a contradiction.

Now consider the case where $q_2 < q$; that is, $q_2 = q - x$, where $x \in \mathbb{N}$
and $x \geq 1$. We also have that $r_2 = r + y$, where $y \in \mathbb{Z}$. Then
we have that
\[ a = bq_2 + r_2 \]
\[ a = b(q - x) + (r + y) \]
\[ a = bq - bx + (r + y) \]
We have that $y = bx \geq b$, which means that $(r + y) = r_2 \geq b$, since
$0 \leq r$. But this is a contradiction.

So in either case we obtain a contradiction. Thus, we can conclude that the pair
$q,r$ is unique.
%%%%%%%%%%%%%%%%%%%%%%%%%%%%%% Problem 3
\section*{Problem 3}
\begin{description}
    \item[(a)]
        Let $c_k$ be the greater of $a_k$ and $b_k$ (if they are equal,
        just let it be $b_k$); that is, \\
        $c_k = (a_k > b_k) \hspace{2mm} ? \hspace{2mm} a_k \hspace{2mm} :
        \hspace{2mm} b_k$. Then we have that
        \[ LCM(p_1^{a_1}p_2^{a_2}...p_k^{a_k}, p_1^{b_1}p_2^{b_2}...
        p_k^{b_k}) = p_1^{c_1}p_2^{c_2}...p_k^{c_k} \]
        We find the $LCM$ this way because the $LCM$ must contain the prime
        factorization of both numbers (clearly this is a multiple of both
        numbers). Any more than that and it will not be
        the least common multiple, and any less and it will not be a multiple of
        both numbers.

        Let $d_k$ be the lesser of $a_k$ and $b_k$ (if they are equal,
        just let it be $b_k$); that is, \\
        $d_k = (a_k < b_k) \hspace{2mm} ? \hspace{2mm} a_k \hspace{2mm} :
        \hspace{2mm} b_k$. Then we have that
        \[ GCD(p_1^{a_1}p_2^{a_2}...p_k^{a_k}, p_1^{b_1}p_2^{b_2}...
        p_k^{b_k}) = p_1^{d_1}p_2^{d_2}...p_k^{d_k} \]
        We find the $GCD$ this way because the $GCD$ must contain the
        smaller part of the prime factorization of both numbers (clearly
        this will divide both numbers). Any more than that and it will
        not divide the smaller number, and any less and it will not be the
        greatest common divisor.
    \item[(b)]
        Let $p_1,p_2,...,p_k$ and $a_1,a_2,...,a_k$ and $b_1,b_2,...,b_k$ be defined
        as in part $(a)$.
        Consider the prime factorization $p_1^{a_1}p_2^{a_2}...p_k^{a_k}$
        of $a$, and the prime factorization $p_1^{b_1}p_2^{b_2}...
        p_k^{b_k}$ of $b$.
        Given our definition of $c_k$ and $d_k$ from above, it is clear that
        $c_k + d_k = a_k + b_k$ for all $k$. Then we have the following:
        \[ LCM(a,b) \cdot GCD(a,b) = p_1^{c_1}p_2^{c_2}...p_k^{c_k} \cdot
        p_1^{d_1}p_2^{d_2}...p_k^{d_k} \]
        \[ LCM(a,b) \cdot GCD(a,b) = p_1^{c_1 + d_1}p_2^{c_2 + d_2}...p_k^{c_k +
        d_k} \]
        \[ LCM(a,b) \cdot GCD(a,b) = p_1^{a_1 + b_1}p_2^{a_2 + b_2}...p_k^{a_k +
        b_k} \]
        \[ LCM(a,b) \cdot GCD(a,b) = a \cdot b \]
\end{description}
%%%%%%%%%%%%%%%%%%%%%%%%%%%%%% Problem 4
\section*{Problem 4}
\begin{description}
    \item[(a)] We can see that, for $0 \leq |a| \leq 7$, we have $a^2
        \equiv r_a \bmod 8$. Our work below shows this:
        \[ 0^2 \equiv 0 \bmod 8 \]
        \[ 1^2 \equiv 1 \bmod 8 \]
        \[ 2^2 \equiv 4 \bmod 8 \]
        \[ 3^2 \equiv 1 \bmod 8 \]
        \[ 4^2 \equiv 0 \bmod 8 \]
        \[ 5^2 \equiv 1 \bmod 8 \]
        \[ 6^2 \equiv 4 \bmod 8 \]
        \[ 7^2 \equiv 1 \bmod 8 \]
        For $|a| > 7$, $k^2 \equiv a^2 \bmod 8$, for some $0 \leq |k| \leq 7$.
        This is because $k \equiv a \bmod 8$ and the fact that if $a \equiv b \bmod
        m$ and $c \equiv d \bmod m$ then $ac \equiv bd \bmod m$ (congruency and
        multiplication). Then, because congruency is transitive, we have
        that for $|a| > 7$, $a^2 \equiv r_a \bmod 8$. So we have proved
        that for any $a \in \mathbb{Z}$, we have $a^2 \equiv r_a \bmod 8$.
    \item[(b)] We have that $1003456789 \bmod 8 \equiv 5$. $5$ is not equivalent to
        $0$, $1$, or $4$. But part $(a)$ says that for any $a \in \mathbb{Z}$,
        we have $a^2 \equiv r_a \bmod 8$, where $r_a$ is either $0$, $1$, or $4$.
        Thus we can conclude that $1003456789$ is not a perfect square.
    \item[(c)] Assume to the contrary that a number of the form
        $3^n + 3^m + 1$, where $n,m \in \mathbb{N}$, is a perfect square.
        For this problem we will do everything under $\bmod 8$.
        By part $(a)$, this means we have
        \[ 3^n + 3^m + 1 \equiv 3^n \bmod 8 + 3^m \bmod 8 + 1 \bmod 8 \equiv
            r_a \]
        We will now show that $3^n \bmod 8$, where $n \in \mathbb{N}$, is
        always equivalent to either $1$ or $3$. To do this we will use induction.
        Our base case is when $n=0$. Then we have $1 \bmod 8 \equiv 1$, and our
        claim is satisified. Next comes our inductive assumption. We will
        assume that $3^n \bmod 8$ is equivalent to either $1$ or $3$ for some
        natural number
        $n > 0$. Now we must show that $3^{n+1} \bmod 8$ is equivalent
        to either $1$ or $3$.  To see this is true, consider the following:
        \[ 3^{n+1} \bmod 8 \equiv 3 \cdot 3^n \bmod 8 \]
        We then have that either
        \[ 3^{n+1} \bmod 8 \equiv 9 \equiv 1 \]
        or
        \[ 3^{n+1} \bmod 8 \equiv 3 \]
        Thus, by induction, we have that $3^n \bmod 8$, where $n \in \mathbb{N}$,
        is always equivalent to either $1$ or $3$. Using this, we have
        three cases:
        \[ 3^n + 3^m + 1 \equiv 1+1+1 \equiv 3 \equiv r_a \]
        \[ 3^n + 3^m + 1 \equiv 1+3+1 \equiv 5 \equiv r_a \]
        \[ 3^n + 3^m + 1 \equiv 3+3+1 \equiv 7 \equiv r_a \]
        All such cases, given $r_a$, are contradictions. Thus we can conclude
        that no number of the form $3^n + 3^m + 1$, where $n,m \in \mathbb{N}$,
        is a perfect square.
\end{description}
%%%%%%%%%%%%%%%%%%%%%%%%%%%%%% Problem 5
\section*{Problem 5}
\[ p=2161 \text{ and } q=3989 \]
\[ n=pq=8620229 \]
\[ \varphi(pq) = n - p - q + 1  = 8614080 \]
\[ e = 6949 \]
Now we will use the extended Euclidean algorithm to find
$de \equiv 1 \bmod \varphi(n)$.

\begin{table}[h]
    \centering
    \begin{tabular}{ l l l }
        8614080 & 1 & 0 \\
        6949 & 0 & 1 \\
        4269 & 1 & -1239 \\
        2680 & -1 & 1240 \\
        1589 & 2 & -2479 \\
        1091 & -3 & 3719 \\
        498 & 5 & -6198 \\
        95 & -13 & 16115 \\
        23 & 70 & -86773 \\
        3 & -293 & 363207 \\
        2 & 2121 & -2629222 \\
        1 & -2414 & 2992429 \\
    \end{tabular}
\end{table}

Then we have that letting $d = 2992429$ gives us
$de \equiv 1 \bmod \varphi(n)$.

Now we can encrypt $m = 2014$.
\[ X =  m^e \bmod n \]
\[ X = 2014^{6949} \bmod 8620229 \]
\[ X = 7986909 \]

Decrypting this $X$ would go as follows.
\[ X^d \bmod n = 7986909^{2992429} \bmod 8620229 \]
\[ X^d \bmod n = 2014 \]
\end{document}

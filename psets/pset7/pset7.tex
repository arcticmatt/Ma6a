%%%%%%%%%%%%%%%%%%%%%%%%%%%%%% Preamble
\documentclass{article}
\usepackage{amsmath,amssymb,amsthm,fullpage}
\usepackage[a4paper,bindingoffset=0in,left=1in,right=1in,top=1in,
bottom=1in,footskip=0in]{geometry}
\newtheorem*{prop}{Proposition}
%\newcounter{Examplecount}
%\setcounter{Examplecount}{0}
\newenvironment{discussion}{\noindent Discussion.}{}
\setlength{\headheight}{12pt}
\setlength{\headsep}{10pt}
\usepackage{fancyhdr}
\pagestyle{fancy}
\fancyhf{}
\lhead{Ma6a Pset 7}
\rhead{Matt Lim}
\pagenumbering{gobble}
\pagenumbering{gobble}
\DeclareMathOperator{\lcm}{lcm}
\begin{document}

%%%%%%%%%%%%%%%%%%%%%%%%%%%%%% Problem 1
\section*{Problem 1}
\begin{description}
    \item[(i)] We have that this is true. That is, we have that the cyclic
        group $C_m$ is commutative. To prove this, we will consider two
        arbitrary $x,y \in G$. Now we must show that $xy = yx$. Since
        $C_m$ is a cyclie group, it has a generator $g$, and we can write
        $x = g^k$ and $y = g^j$. Then, we have that
        \[ xy = g^k g^j \]
        \[ yx = g^j g^k \]
        But clearly, this means that
        \[ xy = g^{k+j} \]
        \[ yx = g^{j+k} \]
        since we are just applying the group operation to the same thing
        (the generator) multiple times (it was also given in lecture 18 that
        we can do this with powers). Then, it follows that
        \[ xy = g^{k+j} = g^{j+k} = yx \]
        Note that we assumed here that $k+j$ is less than $m$. If this is not
        true, then we just take the modulus of it (with respect to $m$),
        and both sides are still equal. Thus, we have proved the desired statement.
    \item[(ii)] This is false. We can give a counterexample. Let
        $x$ be rotation 90 and let $y$ be vertical flip. Consider a starting
        square

        \begin{tabular}{ | l | c | }
            \hline
            A & B \\ \hline
            C & D \\
            \hline
        \end{tabular}

        Let us first consider what happens when we apply $x$, then $y$.
        Applying $x$ (rotation 90) gives us

        \begin{tabular}{ | l | c | }
            \hline
            C & A \\ \hline
            D & B \\
            \hline
        \end{tabular}

        Then, applying $y$ (vertical flip) gives us

        \begin{tabular}{ | l | c | }
            \hline
            D & B \\ \hline
            C & A \\
            \hline
        \end{tabular}

        Now let us consider what happens when we apply $y$, then $x$, to the
        starting square. Applying $y$ (vertical flip) gives us

        \begin{tabular}{ | l | c | }
            \hline
            C & D \\ \hline
            A & B \\
            \hline
        \end{tabular}

        Then, applying $x$ (rotation 90) gives us

        \begin{tabular}{ | l | c | }
            \hline
            A & C \\ \hline
            B & D \\
            \hline
        \end{tabular}

        So clearly applying $x$ then $y$ and applying $y$ then $x$ results
        in different squares. So for this example, $xy \neq yx$. So we have
        found a counterexample and shown that the group of symmetries of the
        square is not commutative.
    \item[(iii)] This is true. Let $G$'s operation be represented as $*$.
        We have that $G$ satisfies $(ab)^2 = a^2b^2$
        for every $a,b \in G$. This is the same as saying that $G$ satisfies
        $abab = aabb$ for every $a,b \in G$. So, we have that
        $abab = aabb$. Then we have that $abab = aabb \implies bab = abb$. This
        is because $bab \neq abb \implies abab \neq aabb$, since to go from
        the left side of the implies to the right we are just applying the
        operation $*$ to the same thing ($a$) and two
        unequal things ($bab$ and $abb$). And doing this cannot possibly turn
        it into an equality. Then we have that
        $bab = abb \implies ba = ab$. This is because $ba \neq ab \implies
        bab \neq abb$, since to go from the left side of the implies to the right
        we are just applying the operation $*$ to the
        same thing ($b$) and two unequal things ($ba$ and $ab$). And
        doing this cannot possibly turn it into an equality. So now we have
        that $G$ satisfies $ba = ab$ for every $a,b \in G$. But this means that
        $G$ is commutative. So we have proved that any group $G$ that
        satisfies $(ab)^2 = a^2b^2$ for every $a,b \in G$ is commutative.
\end{description}

%%%%%%%%%%%%%%%%%%%%%%%%%%%%%% Problem 2
\section*{Problem 2}
We have that $\gcd (m,n) \neq 1$, and we wish to prove that $C_m \times C_n$
is not cyclic. We have that we can write
\[ C_m = \{1, g, \cdots, g^{m-1} \} \]
\[ C_n = \{1, h, \cdots, h^{n-1} \} \]
Now, to show that $C_m \times C_n$ is not cyclic, it suffices to show that
$C_m \times C_n$ does not have a generator. So, WLOG, we can consider any
element from $(a,b)$ from $C_m \times C_n$. Consider $(a,b)^{mn}$. We have
that this takes $a$ through $n$ cycles and $b$ through $m$ cycles, since $m$
divides $mn$ and $n$ divides $mn$. Now consider $k < mn$, where $m$ divides $k$
and $n$ divides $k$. Let $k/m = i$ and $k/n = j$ (since $m$ and $n$ divide $k$,
$i$ and $j$ are whole numbers). We have that $k$ exists
because $\gcd (m,n) \neq 1$ (more specifically, we can pick $k$ to be
$\lcm (m,n) = mn / \gcd(m,n))$. Now consider $(a,b)^{k}$. We have that
this takes $a$ through $k/m = i$ cycles and $b$ throught $k/n = j$ cycles.
Now we have that $(a,b)^{mn} = (a,b)^{k}$, since $a$ going through $\alpha$ cycles,
where $\alpha$ is some whole number, is the same as $a$ going through
$\beta$ cycles, where $\beta$ is some other whole number (and the same holds
for $b$). This basically means that no matter what element we try to pick
as our generator, we will repeat elements when we raise the generator
to $k$ and to $mn$. But since we have these repeat elements, we cannot generate
the entire group. So we have that no element $(a,b)$ from $C_m \times C_n$ can
be a generator, and thus that $C_m \times C_n$ is not cyclic.
%%%%%%%%%%%%%%%%%%%%%%%%%%%%%% Problem 3
\section*{Problem 3}
\begin{description}
    \item[(a)] We will prove this. Consider all elements $a \in G$ where
        $|a| > 2$. Then we have that $a^{-1} \neq a$ and that $|a^{-1}| > 2$.
        To see this, consider an arbitrary $a$ where $|a| > 2$. Let $b = a^{-1}$,
        the unique inverse for $a$ (the fact that such a unique inverse exists
        was proved in class 18).
        We have that $a^k = e$ and that $ab = e$, where $k = |a|$.
        So then we have that
        $a^kb = a^{k-1} \implies eb = a^{k-1} \implies b = a^{k-1}$. Then,
        since $b$ is just a power of $a$, it follows that if $|a| > 2$, then
        $|b| > 2$ (in fact, the orders will be equal). This gives us that
        elements in $G$ that have order greater than $2$ come in pairs. Then
        we have that $G$ must contain the identity $e$, which clearly has
        an order less than $2$. So overall, we have that of all the elements in
        the group, an odd number of them are not of order $2$. This means
        that an odd number of the elements are of order $2$. So we are done.
    \item[(b)] We have that the number of symmetries of the regular hexagon
        is $2*6 = 12$. We can see this in the following way. There is no action,
        and $5$ rotations. Then there are $6$ flips (we can flip along lines
        drawn from corner to corner and along lines drawn from midpoint to
        middpoint). So the order for the group of symmetries of the regular
        hexagon is $12$. Then we have that the order for the alternating
        group $A_4$ is $3$. We have that this is true because the order
        of $A_n$ is half the order of $S_n$ (class 20), and the order of $S_n$
        is just $n!$. So here, $|A_4| = \frac{4!}{2} = 12$. So the order of
        our groups are the same. However, we have that $A_4$ only contains
        permutations from $S_n$. So basically, permuations of the form $(12)(34)$
        or $(123)$, and so on. With this being true, it is clear that there
        cannot exist an isomorphism between these two groups, because we
        cannot clearly map from the permutations to the hexagon symmetries and
        back. In order to do so, we would need permutations from $S_6$, because
        then we would be able to map every edge to another edge using
        such permutations. However, as it is, we do not have complex enough
        permutations to sufficiently map back and forth between the two groups.
        Shown below is an example of why there cannot be an isomorphism.
\newpage
\end{description}

%%%%%%%%%%%%%%%%%%%%%%%%%%%%%% Problem 4
\section*{Problem 4}
Recall the bound on the size of the harmonic series
\[ \sum_{n=1}^{\infty} \frac{1}{n} \leq \ln(n) + 1 \]
Now we want to show that for every $n$, we have $\bar{t}(n) \leq \ln(n) + 1$.
Given the bound on the size of the harmonic series, we can instead show that
for every $n$, the following bound holds:
\[ \bar{t}(n) = \frac{1}{n} \sum_{j=1}^n t(j) \leq \sum_{j=1}^n \frac{1}{j} \]
Which is equivalent to the following:
\[ \sum_{j=1}^n t(j) \leq n \sum_{j=1}^n \frac{1}{j} \]
To show this inequality, we can use double counting. So we will make a table
as follows:

\vspace{2mm}
\begin{tabular}{ | l | c | c | c | c | c | c |}
    \hline
         & 1 & 2 & 3 & 4 & ... & n \\ \hline
    s(1) & x &   &   &   &     &   \\ \hline
    s(2) & x & x &   &   &     &   \\ \hline
    s(3) & x &   & x &   &     &   \\ \hline
    s(4) & x & x &   & x &     &   \\ \hline
    ...  &   &   &   &   &     &   \\ \hline
    s(n) & x & ? & ? & ? &     & x \\ \hline
\end{tabular}
\vspace{2mm}

\noindent Here, $s(i)$ denotes the set of elements in $\{1,2,3,\cdots,i\}$ that
divide $i$, and there is a mark in a box $(i,j)$ if $s(i)$ contains the number
$j$. So the number of marks in a given row $i$ is equivalent to $t(i)$, which
means that the number of marks total in the table is just equal to
$\sum_{j=1}^n t(j)$.
Now, notice that for each column $j$, we have that the number of marks in that
column is bounded above by $\frac{n}{j}$. That is equivalent to saying that the
number of numbers from $1$ to $n$ that are divisible by $j$ is bounded above by
$\frac{n}{j}$. This is true for the following reason.
Consider the numbers $1, 2, 3, \cdots, \lfloor \frac{n}{j} \rfloor$.
Multiplying each of these numbers by $j$ gives us all the numbers that are
divisible by $j$ and that range from $1$ to $n$. And the cardinality of that set of
numbers is bounded above by $\frac{n}{j}$. So clearly, if we had more than
$\frac{n}{j}$ numbers from $1$ to $n$ that were divisible by $j$, we would have
a contradiction. Now that we have that the number of marks in each column
$j$ is bounded above by $\frac{n}{j}$, we can get an upper bound on the number
of marks in the table as follows:
\[ \sum_{j=1}^n t(j) \leq \sum_{j=1}^n \frac{n}{j} = n\sum_{j=1}^n \frac{1}{j} \]
We got this sum by just summing along the columns, and using the simple fact
that summing the number of marks in all the columns just gives us the total
number of marks in the table. Then we have that
this is exactly the bound that we wanted. So, we can conclude that for every
$n$, we have $\bar{t}(n) \leq \ln(n) + 1$.

%%%%%%%%%%%%%%%%%%%%%%%%%%%%%% Problem 5
\section*{Problem 5}
Note: all congruences in this problem will be mod $a$.
So, we have that $a$ is a composite number which is not a Carmichael number,
and that we are choosing a $q \in \{1,2,3,\cdots,a-1\}$.
We can break this problem down into two.

The first case is when $\gcd (a,q) = g \neq 1$. In this case, the test always
finds that $a$ is composite because $q^{a-1} \not\equiv 1$ will always be
true (which is right, because $\gcd(a,q) \neq 1$). We have that this is the
case because $q^{a-1} = gx$, where $x$ is an integer. We also have that
$a = gy$, where $y$ is an integer. So clearly, $gx \not\equiv 1 \mod gy$.

The second case is when $\gcd (a,q) = 1$. In this case, since $a$ is not
a Carmichael number, we have that there exists at least one $q$ such that
\[ q^{a-1} \not\equiv 1 \]
We will call this $q$ $q_0$. We also have a set of $q$s such that
\[ q^{a-1} \equiv 1 \]
We will call this set $A$. This set will be under mod $a$. Then we have that
\[ A = \{q \ | \  q^{a-1} \equiv 1 \} \]
Let $|A| = n$.
Now we want to show that we can generate a set $B$ of the form
\[ B = \{q \ | \ q^{a-1} \not\equiv 1 \} \]
such that $|B| \geq n$. To do this, consider the set
\[ B = \{ q_0q_1, q_0q_2, \cdots, q_0q_n \} \]
where $q_1, \cdots, q_n \in A$. This set will also be under mod $a$.
Now, consider an arbitrary element
$q_0q_i$ of $B$. We want to show that $(q_0q_i)^{a-1} \not\equiv 1$. So,
since we have that $q_i^{a-1} \equiv 1$, we get the following:
\[ (q_0q_i)^{a-1} \equiv q_0^{a-1}q_i^{a-1} \equiv q_0^{a-1} \not\equiv 1 \]
Now we just need to show that each element in $B$ is unique. To show this,
assume to the contrary that $q_0q_i \equiv q_0q_j$. Now notice that,
since $\gcd (a, q_0) = 1$, we have that there exists a modular multiplicative
inverse $q_0^{-1}$ such that $q_0q_0^{-1} \equiv 1$. Thus we can multiply
both sides of $q_0q_i \equiv q_0q_j$ by $q_0^{-1}$ to obtain
$q_i \equiv q_j$. But this means that whenever two elements of $B$ are the
same, the elements of $A$ are the same, which means that $i=j$ since $A$
is a set under mod $a$. Then, given
how we constructed $B$, this means that whenever two elements of $B$ are the
same, those two elements must be the exact same element. So we have shown
that when $\gcd (a,q) = 1$, there exists a set of unique false witnesses $A$ of
size $n$, and there exists a set of unique witnesses $B$ of at least size $n$.

Now we can put our two cases together. First we have that there exists
an $\epsilon$ probability of getting $\gcd(a,q) \neq 1$. In this case the
test succeeds. Then we have that, in the case that this doesn't happen, there
is at most a $1/2$ probability of the test failing (because we have at least
as many elements in $B$, which are witnesses, as in $A$, which are false witnesses).
This is the probability of the second case failing. So, more precisely, the
probability that the
second case fails is $(1/2) \dot (1 - \epsilon) = 1/2 - \epsilon/2$.
So, putting these two cases together (which are clearly disjoint) we have
that the probability that the test fails to discover that $a$ is composite
when using a uniformly chosen $q \in \{1,2,3,\cdots,a-1\}$ is bounded above by
\[ \frac{1}{2} - \frac{\epsilon}{2} \]
\end{document}

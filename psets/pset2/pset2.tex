%%%%%%%%%%%%%%%%%%%%%%%%%%%%%% Preamble
\documentclass{article}
\usepackage{amsmath,amssymb,amsthm,fullpage}
\usepackage[a4paper,bindingoffset=0in,left=.8in,right=.8in,top=.7in,
bottom=1in,footskip=0in]{geometry}
\newtheorem*{prop}{Proposition}
%\newcounter{Examplecount}
%\setcounter{Examplecount}{0}
\newenvironment{discussion}{\noindent Discussion.}{}
\setlength{\headheight}{12pt}
\setlength{\headsep}{10pt}
\usepackage{fancyhdr}
\pagestyle{fancy}
\fancyhf{}
\lhead{Ma6a Pset 2}
\rhead{Matt Lim}
\pagenumbering{gobble}
\pagenumbering{gobble}
\begin{document}
%%%%%%%%%%%%%%%%%%%%%%%%%%%%%% Problem 1
\section*{Problem 1}
Let
\[ x = 2793 \]
\[ y = 1467 \]
We will first find a composite witness for $x = 2793$. We have that $x - 1 = 2792 =
2^3 \cdot 349$. This gives us $s = 2$ and $d = 349$. We will choose our witness to be
$a = 2$. Now we must verify (for equivalency, we are using $\bmod$ 2793).
\[ a^d = 2^{349} \equiv 1724 \not\equiv 1 \]
For r = 0, we get the following
\[ a^{2^rd} = a^d \equiv 1724 \not\equiv -1 \]
For r = 1, we get the following
\[ a^{2^rd} = a^{2d} \equiv 424 \not\equiv -1 \]
For r = 2, we get the following
\[ a^{2^rd} = a^{4d} \equiv 1024 \not\equiv -1 \]
So we have that $a = 2$ is a composite witness for 2793.


\vspace{5mm}
We will now find a composite witness for $y = 1467$. We have that $y - 1 = 1466 =
2 * 733$. This gives us $s = 1$ and $d = 733$. We will choose our witness to be
$a = 2$. Now we must verify (for equivalency, we are using $\bmod$ 1467).
\[ a^d = 2^{733} \equiv 1451 \not\equiv 1 \]
For r = 0, we get the following
\[ a^d = 2^{733} \equiv 1451 \not\equiv -1 \]
So we have that $a = 2$ is a composite witness for 1467.

%%%%%%%%%%%%%%%%%%%%%%%%%%%%%% Problem 2
\section*{Problem 2}
\begin{description}
    \item[(a)] We are trying to find $\varphi(p^s)$, where $p$ is a prime number
        and $s \in \mathbb{N} \backslash \{0\}$. We can see that the answer is
        \[ \varphi(p^s) = p^s - p^{s-1} \]
        This is because there are $p^{s-1}$ numbers that share a common factor
        with $p^s$. These factors are $1 \cdot p, 2 \cdot p, 3 \cdot p, ...,
        p^{s-1} \cdot p$.
    \item[(b)]
        Given a number $n$, we have that
        \[ \varphi(n) = n \cdot \prod_{p|n}(1 - \frac{1}{p}) \]
        So given $ab$, we have that
        \[ \varphi(ab) = ab \cdot \prod_{p|ab}(a - \frac{1}{p}) \]
        But $a$ and $b$ are relatively prime, which means that for every $p$,
        $p$ can only have a common factor with only one of $a$ or $b$ (never
        both). So because of this we have that
        \[ \varphi(ab) = a \cdot \prod_{p|a}(1 - \frac{1}{p}) \cdot b \cdot
        \prod_{q|b}(1 - \frac{1}{q}) \]
        But this is just equivalent to
        \[ \varphi(ab) = \varphi(a) \cdot \varphi(b) \]
        So we have that $\varphi(ab) = \varphi(a) \cdot \varphi(b)$ when $a$
        and $b$ are relatively prime.
    \item[(c)]
        We have that all the numbers we are multiplying together ($p_1^{s_1},
        p_2^{s_2}, \cdots, p_k^{s_k}$) are relatively
        prime, that $p_1, p_2, \cdots, p_k$ are prime numbers, and that
        $s_1, s_2, \cdots, s_k \in \mathbb{N} \backslash \{0\}$.
        Thus, using parts (a) and (b), we have that the answer is simply
        \[ \varphi(p_1^{s_1}p_2^{s_2} \cdots p_k^{s_k}) = \varphi(p_1^{s_1})
        \varphi(p_2^{s_2}) \cdots \varphi(p_k^{s_k}) \]
        \[ \varphi(p_1^{s_1}p_2^{s_2} \cdots p_k^{s_k}) = (p_1^{s_1}-p_1^{s_1-1})
        (p_2^{s_2}-p_2^{s_2-1}) \cdots (p_k^{s_k}-p_k^{s_k-1}) \]
\end{description}

%%%%%%%%%%%%%%%%%%%%%%%%%%%%%% Problem 3
\section*{Problem 3}
Writing $(x-y)^n$ using binomial coefficients, we get
\[ (x-y)^n = \binom{n}{0}x^n - \binom{n}{1}x^{n-1}y^1 + \binom{n}{2}x^{n-2}y^2
- \binom{n}{3}x^{n-3}y^3 + \binom{n}{4}x^{n-4}y^4 + \cdots + (-1)^n \cdot
\binom{n}{n}x^0y^n \]
Assigning $x=1$ and $y=-1$, we get
\[ (x-y)^n = \binom{n}{0} + \binom{n}{1} + \binom{n}{2} + \binom{n}{3}
    + \binom{n}{4} + \cdots + \binom{n}{n} \]
For any such sequence of binomial coefficients $\binom{n}{0} + \binom{n}{1} +
\binom{n}{2} + \binom{n}{3} + \binom{n}{4} + \cdots + \binom{n}{n}$ as we have
on the right hand side, we have that $\binom{n}{0} + \binom{n}{2} + \binom{n}{4}
+ \cdots$ divides it by 2. We can see this is true by induction. Take $n=2$
to be the base case. Then we have that the binomial coefficient series for this
is $\binom{2}{0} + \binom{2}{1} + \binom{2}{2}$. We can see that $\binom{2}{0}
+ \binom{2}{2} = 2 = \binom{2}{1}$. So the base case is satisfied. Now assume
that our statement is true for some $n$. Now we want to show it is true for
$n+1$. So consider
\[ \binom{n+1}{0} + \binom{n+1}{1} + \binom{n+1}{2} + \binom{n+1}{3}
    + \binom{n+1}{4} + \cdots + \binom{n+1}{n+1} \]
This becomes
\[ \left[\binom{n}{-1} + \binom{n}{0}\right] + \left[\binom{n}{0} +
    \binom{n}{1}\right] + \left[\binom{n}{1} + \binom{n}{2}\right] +
    \left[\binom{n}{2} + \binom{n}{3}\right] + \cdots + \left[\binom{n}{n}
    + \binom{n}{n+1}\right] \]
by Pascal's rule, which is equivalent to
\[ 2 \left[\binom{n}{0} + \binom{n}{1} + \binom{n}{2} + \cdots + \binom{n}{n}\right]
\]
So half of this is just
\[ \binom{n}{0} + \binom{n}{1} + \binom{n}{2} + \cdots + \binom{n}{n} \]
which is just
\[ \binom{n+1}{0}  + \binom{n+1}{2} + \binom{n+1}{4} + \cdots \]
So we have proved our statement by induction.
Finally, since we chose $x=1$, $y=-1$, we can conclude that
\[ (x-y)^n = 2^n = 2\left[\binom{n}{0} + \binom{n}{2} + \binom{n}{4} +
    \cdots\right] \]


%%%%%%%%%%%%%%%%%%%%%%%%%%%%%% Problem 4
\section*{Problem 4}
So we have a set of numbers $\{1,2,3,...,n\}$. From this set we will take the set
$\{1,2,3,..., n-k+1\}$. Then from this set will will take sets of $k$ numbers
$\{a_1, a_2, a_3, ..., a_k\}$. Then we will map these sets to sets like so:
$\{a_1, a_2 + 1, a_3 + 2, ..., a_k + k - 1\}$. We clearly have that this mapping
gets rid of consecutive elemnents. We also have that this mapping creates subsets
of $\{1,2,3,\cdots,n\}$, since the largest number in the set we were mapping
from was $n-k+1$. Finally, we have that this mapping is exhaustive in
creating subsets of $\{1,2,3,\cdots,n\}$ of size $k$ that do not contain
two consecutive elements, since given any arbitrary such subset, our mapping
can clearly create it. Another way to see this is that if we have
a subset of $\{1,2,3,\cdots,n\}$ of size $k$ that does not contain two
consecutive elements, we can get back to a subset of $\{1,2,3,\cdots,n-k+1\}$
by applying the reverse of our transformation (and vice versa).
So overall, the number of subsets of size $k$ that do not contain two
consecutive elements is $\binom{n-k+1}{k}$, since our mapping works on all
$k$ size subsets from the set $\{1,2,3,\cdots,n-k+1\}$.

%%%%%%%%%%%%%%%%%%%%%%%%%%%%%% Problem 5
\section*{Problem 5}
Given $k_i = f(i) - f(i-1)$, $k_0 = f(0)$, and $k_{n+1} = n - f(n)$, we have that
$\sum_{i=0}^{n+1} k_i = n$. We can see that this is analogous to dividing $n$ balls
up into $n+2$ bins, where we can place any number
of balls in each bin. This is just $\binom{n+n+2-1}{n+2-1} = \binom{2n+1}{n+1}$. That
is, there are $\binom{2n+1}{n+1}$ ways to put $n$ balls into $n+2$ bins, where
here our bins represent $k_i$'s. In other words, there are $\binom{2n+1}{n+1}$ ways
to choose our $k_i$ values, where all the $k_i$ values will be non-negative (since
we can put at the very least 0 balls in each "bin"). And choosing a set of
non-negative $k_i$ values is the same as choosing a monotonically increasing
function, given how we defined our $k_i$s. So we have that there are
$\binom{2n+1}{n+1}$ monotonically increasing functions.
\end{document}

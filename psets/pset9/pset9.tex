%%%%%%%%%%%%%%%%%%%%%%%%%%%%%% Preamble
\documentclass{article}
\usepackage{amsmath,amssymb,amsthm,fullpage}
\usepackage[a4paper,bindingoffset=0in,left=1in,right=1in,top=1in,
bottom=1in,footskip=0in]{geometry}
\newtheorem*{prop}{Proposition}
%\newcounter{Examplecount}
%\setcounter{Examplecount}{0}
\newenvironment{discussion}{\noindent Discussion.}{}
\setlength{\headheight}{12pt}
\setlength{\headsep}{10pt}
\usepackage{fancyhdr}
\pagestyle{fancy}
\fancyhf{}
\lhead{Ma6a Pset 9}
\rhead{Matt Lim}
\pagenumbering{gobble}
\begin{document}

%%%%%%%%%%%%%%%%%%%%%%%%%%%%%% Problem 1
\section*{Problem 1}
Let
\[ a_n = \frac{1}{2} \Big[ (1 + \sqrt2)^n + (1 - \sqrt2)^n \Big] \]
We can see that
\[ a_0 = 1 \]
\[ a_1 = 1 \]
\[ a_2 = 3 \]
\[ a_3 = 7 \]
\[ a_4 = 17 \]
\[ a_5 = 41 \]
\[ a_6 = 99 \]
\[ a_7 = 239 \]
Observing this leads us to believe that the recurrence relation for $a_n$
is the following:
\[ a_{i+2} = 2a_{i+1} + a_i, i \geq 0 \]
We can prove that this is the correct recurrence relation for $i \geq 0$
by plugging into the generating function and seeing what results, as follows.
\[ A(x) = a_0 + a_1x + a_2x^2 + a_3x^3 + a_4x^4 + \cdots \]
\[ A(x) = a_0 + a_1x + (2a_1 + a_0)x^2 + (2a_2 + a_1)x^3 + (2a_3 + a_2)x^4
    + \cdots \]
\[ A(x) = a_0 + a_1x + 2x(A(x) - a_0) + x^2A(x) \]
\[ A(x) = 1 + x + 2x(A(x) - 1) + x^2A(x) \]
\[ x^2A(x) + 2xA(x) - A(x) = x - 1 \]
\[ A(x) = \frac{x-1}{x^2 + 2x - 1} \]
\[ A(x) = \frac{1}{2} \Big( \frac{\sqrt2 + 1}{x + \sqrt2 + 1} \Big)
    + \frac{1}{2} \Big( \frac{\sqrt2 - 1}{x - \sqrt2 + 1} \Big) \]
\[ A(x) = \frac{1}{2} \Big( \frac{1}{1 - (-\frac{x}{\sqrt2 + 1})} \Big)
    + \frac{1}{2} \Big( \frac{1}{1 - (-\frac{x}{1 - \sqrt2})} \Big) \]
\[ A(x) = \frac{1}{2} \sum_{n \geq 0} \Big(-\frac{1}{\sqrt2 + 1} \Big)^n x^n
    + \frac{1}{2} \sum_{n \geq 0} \Big(-\frac{1}{1 - \sqrt2} \Big)^n x^n \]
\[ A(x) = \frac{1}{2} \sum_{n \geq 0} \Big[ \Big(-\frac{1}{\sqrt2 + 1} \Big)^n
    + \Big(-\frac{1}{1 - \sqrt2} \Big)^n \Big] x^n \]
\[ A(x) = \frac{1}{2} \sum_{n \geq 0} \Big[ \Big(-\frac{\sqrt2 - 1}{2 - 1} \Big)^n
    + \Big(-\frac{1 + \sqrt2}{1 - 2} \Big)^n \Big] x^n \]
\[ A(x) = \frac{1}{2} \sum_{n \geq 0} \Big[ \Big(-\frac{\sqrt2 - 1}{1} \Big)^n
    + \Big(-\frac{1 + \sqrt2}{-1} \Big)^n \Big] x^n \]
\[ A(x) = \frac{1}{2} \sum_{n \geq 0} \Big[ (1 - \sqrt2)^n
    + (1 + \sqrt2)^n \Big] x^n \]
\[ A(x) = \sum_{n \geq 0} \frac{1}{2} \Big[ (1 + \sqrt2)^n
    + (1 - \sqrt2)^n \Big] x^n \]
We can see that our $A(x)$ gives us the same coefficients $a_n$
as the problem gives us. Thus we have that our recurrence relation
$a_{i+2} = 2a_{i+1} + a_i$ is correct. Alternatively, we can prove
that our recurrence relation is right for $i \geq 0$ in the following way:
We can check the base case, $i = 0$, by checking that
$a_2 = 2a_1 + a_0$. We have that $a_2 = 3$, $a_1 = 1$, and $a_0 = 1$. So our
base case is satisfied. Then we can assume it is true for $i+1$ and prove it
is true for $i+2$ as follows:
\[ 2a_{i+1} + a_i
    = \Big[ (1 + \sqrt2)^{i+1} + (1 - \sqrt2)^{i+1} \Big]
    + \frac{1}{2} \Big[ (1 + \sqrt2)^i + (1 - \sqrt2)^i \Big] \]
\[ 2_{i+1} + a_i = (1 + \sqrt2)^i (1 + \sqrt2 + \frac{1}{2})
    + (1 - \sqrt2)^i (1 - \sqrt2 + \frac{1}{2}) \]
\[ 2_{i+1} + a_i = (1 + \sqrt2)^i \frac{(1 + \sqrt2)^2}{2}
    + (1 - \sqrt2)^i \frac{(1 - \sqrt2)^2}{2} \]
\[ 2_{i+1} + a_i = \frac{1}{2} \Big[ (1 + \sqrt2)^{i+2} + (1 - \sqrt2)^{i+2} \Big] \]
Then we have that
\[ a_{i+2} = \frac{1}{2} \Big[ (1 + \sqrt2)^{i+2} + (1 - \sqrt2)^{i+2} \Big] \]
So then we can see that
\[ a_{i+2} = 2_{i+1} + a_i \]
So we have that our recurrence is true by induction as well.
Then we can see that, by induction and
using our recurrence relation which we have proved to be correct,
every $a_n$ is an integer. The base case is satisfied because we have that
$a_0 = a_1 = 1$. Then, assume using strong induction that for all $0 \leq k \leq n$,
$a_k$ is an integer. Then we have that $a_{n+1}$ is also an integer, because
$a_{n+1} = 2a_{n} + a_{n-1}$. To explain this further, we can see that
$2a_{n}$ and $a_{n-1}$ are integers by our inductive assumption (and because
if we multiply an integer by two, which is also an integer, it is
still an integer). And clearly when we add two integers, the result is
still an integer. Thus we can conclude that $a_n$ is an integer for all $n \geq 0$,
and since
\[ a_n = \frac{1}{2} \Big[ (1 + \sqrt2)^n + (1 - \sqrt2)^n \Big] \]
this proves the desired statement.

%%%%%%%%%%%%%%%%%%%%%%%%%%%%%% Problem 2
\section*{Problem 2}
\begin{description}
    \item[(a)]
        \[ a_0 =1 \]
        \[ a_1 = 3 \]
        \[ a_{i+2} = 4a_{i+1} - 4a_i, i \geq 0 \]
        We can write the generating function as follows:
        \[ A(x) = a_0 + a_1x + a_2x^2 + a_3x^3 + a_4x^4 + a_5x^5 + \cdots \]
        \[ A(x) = a_0 + a_1x + (4a_1 - 4a_0)x^2 + (4a_2 - 4a_1)x^3 +
            (4a_3 - 4a_2)x^4 + \cdots \]
        \[ A(x) = a_0 + a_1x + 4x(A(x) - a_0) - 4x^2A(x) \]
        \[ A(x) = a_0 + a_1x + 4xA(x) - 4xa_0 - 4x^2A(x) \]
        \[ 4x^2A(x) - 4xA(x) + A(x) = a_0 + a_1x - 4xa_0 \]
        \[ A(x)(4x^2 - 4x + 1) = 1 + 3x - 4x \]
        \[ A(x) = \frac{1 - x}{4x^2 - 4x + 1} \]
        \[ A(x) = \frac{1}{2(2x-1)^2} - \frac{1}{2(2x-1)} \]
        \[ A(x) = \frac{1}{2(1-2x)^2} + \frac{1}{2(1-2x)} \]
        \[ \frac{1}{2} \sum_{n \geq 0} \binom{1+n}{n} 2^n x^n
            + \frac{1}{2} \sum_{n \geq 0} 2^n x^n \]
        \[ \frac{1}{2} \sum_{n \geq 0} (n+1) 2^n x^n + \frac{1}{2} \sum_{n \geq 0}
            2^n x^n \]
        \[ \frac{1}{2} \sum_{n \geq 0} (n+2) 2^n x^n \]
        \[ \sum_{n \geq 0} (n+2) 2^{n-1} x^n \]
        So then we have that
        \[ a_n = (n+2)2^{n-1} \]
    \item[(b)]
        \[ c_0 = 1 \]
        \[ c_i = c_0 + c_1 + \cdots + c_{i-1} \text{ or }
                c_i = \sum_{j = 0}^{i - 1} c_j \]
        We can write the generating function as follows:
        \[ C(x) = c_0 + c_1x + c_2x^2 + c_3x^3 + c_4x^4 + \cdots \]
        \[ C(x) = c_0 + c_0x + (c_0 + c_1)x^2 + (c_0 + c_1 + c_2)x^3
            + (c_0 + c_1 + c_2 + c_3)x^4 + \cdots \]
        Now notice that
        \[ C(x)(1 + x + x^2 + x^3 + \cdots) = C(x)\Big(\frac{1}{1-x}\Big) =
            (c_0 \cdot 1) + (c_0 \cdot 1 + c_1 \cdot 1)x + (c_0 \cdot 1
            + c_1 \cdot 1 + c_2 \cdot 1)x^2 + \cdots \]
        Then we can see that, by combining our observation with our previous
        work, we have the following:
        \[ C(x) = c_0 + C(x)\Big(\frac{1}{1-x}\Big)(x) \]
        \[ C(x) = c_0 + C(x)\Big(\frac{x}{1-x}\Big) \]
        \[ C(x)\Big(1 - \frac{x}{1-x}\Big) = 1 \]
        \[ C(x)\Big(\frac{1-2x}{1-x}\Big) = 1 \]
        \[ C(x) = \frac{1-x}{1-2x} \]
        \[ C(x) = \frac{1}{2} - \frac{1}{2}\Big(\frac{1}{2x-1}\Big) \]
        \[ C(x) = \frac{1}{2} + \frac{1}{2}\Big(\frac{1}{1-2x}\Big) \]
        \[ C(x) = \frac{1}{2} + \frac{1}{2} \sum_{n \geq 0} 2^n x^n \]
        \[ C(x) = \frac{1}{2} + \sum_{n \geq 0} 2^{n-1} x^n \]
        So we can see that $c_n = 2^{n-1}, n \geq 1$ and $c_0 = 1$.
\end{description}

%%%%%%%%%%%%%%%%%%%%%%%%%%%%%% Problem 3
\section*{Problem 3}
We can write out the first couple of $b_i$s as follows:
\[ \emptyset \implies b_0 = 1 \]
\[ \emptyset, \{1\} \implies b_1 = 1 \]
\[ \emptyset, \{1\}, \{2\} \implies b_2 = 3 \]
\[ \emptyset, \{1\}, \{2\}, \{3\} \implies b_3 = 4 \]
\[ \emptyset, \{1\}, \{2\}, \{3\}, \{4\}, \{1,4\} \implies b_4 = 6 \]
\[ \emptyset, \{1\}, \{2\}, \{3\}, \{4\}, \{5\}, \{1,4\}, \{1,5\}, \{2,5\}
    \implies b_5 = 9 \]
\[ \emptyset, \{1\}, \{2\}, \{3\}, \{4\}, \{5\}, \{6\}, \{1,4\}, \{1,5\}, \{1,6\},
    \{2,5\}, \{2,6\}, \{3,6\}
    \implies b_6 = 13 \]
\[ \emptyset, \{1\}, \{2\}, \{3\}, \{4\}, \{5\}, \{6\}, \{7\}, \{1,4\}, \{1,5\},
    \{1,6\}, \{1,7\}, \{2,5\}, \{2,6\}, \{2,7\}, \{3,6\}, \{3,7\}, \{4,7\}, \]
\[ \{1,4,7\} \implies b_7 = 19 \]
We can see from this that the recurrence relation is
\[ b_{n+3} = b_{n+2} + b_n \]
Let $R = \{1,2,\cdots, n+3\}$ and $S = \{1,2,\cdots,n\}$.
Then, we can explain this relation in the following way. $b_{n+2}$ gives us the
number of valid subsets of $R$ that do not include the $n+3$ element. Then, if
we append the $n+3$ element to every element in the valid subsets of $S$,
of which there are $b_n$, we get all the valid subsets of $R$ which contain
the $n+3$ element, because every subset in the valid subsets of $S$ has a maximum
element of $n$, and $(n+3) - n = 3 > 2$. In other words, the valid subsets of $S$
are exactly those subsets that we can append $n+3$ to in order to form another
valid subset. So overall we have $b_{n+2} + b_n$ as desired.

\vspace{3mm}
\noindent Then we can write the generating function of $b_n$ in the form
$F(x)/G(x)$ as follows:
\[ B(x) = b_0 + b_1x + b_2x^2 + b_3x^3 + b_4x^4 + b_5x^5 + \cdots \]
\[ B(x) = b_0 + b_1x + b_2x^2 + (b_2 + b_0)x^3 + (b_3 + b_1)x^4 + (b_4 + b_2)x^5
    \cdots \]
\[ B(x) = b_0 + b_1x + b_2x^2 + x^3B(x) + xB(x) - b_1x^2 - b_0x \]
\[ B(x) = 1 + 2x + 3x^2 + x^3B(x) + xB(x) - 2x^2 - x \]
\[ B(x) - x^3B(x) - xB(x) = 1 + x + x^2 \]
\[ B(x) = \frac{x^2 + x + 1}{-x^3 - x + 1} \]

%%%%%%%%%%%%%%%%%%%%%%%%%%%%%% Problem 4
\section*{Problem 4}
We can see that the generating function for $p_k(n)$ is
\[ G = (x + x^2 + x^3 + \cdots)^k \]
We have this because when we multiply this out, for each resulting $x^n$, we have that
the coefficient is the number of ways we can partition $n$ into $k$ parts where
the order of summands does matter. We have that this is true because when
we multiply this out to create terms, $k$ elements get multiplied. And then
the exponents represent sizes of the partitions. So each $x^n$ in the resulting
expansion represents $x^ax^bx^c \cdots x^i$, where $a+b+c+ \cdots + i  = n$.
And we can see that the order of summands here does matter, because we can
have $x^ix^k = x^n$ and $x^kx^i = x^n$ which leads to two $x^n$s and thus
represents two partitions.  In other words, each $x^n$ represents a valid
partition of $n$. So if we have $cx^n$ when we multiply this out, we have
that there are $c$ ways to partition $n$ into $k$ parts where the order
of the summands does matter, because each $x^n$ that we get represents
a unique way that we can partition $n$. Thus this is the generating function.
For an example, if $k = 2$, then we have
\[ (x+x^2+\cdots)(x+x^2+\cdots) \]
\[ x^2 + x^3 + x^3 + \cdots  = x^2 + 2x^3 + \cdots \]
So we can see that $p_2(3) = 2$ ($2+1,1+2)$, $p_2(2) = 1$ ($1+1$), etc.
Then, we can factor out an $x^k$ as follows:
\[ G = (x(1 + x + x^3 + \cdots)^k) \]
\[ G = x^k(1 + x + x^3 + \cdots)^k \]
Remember that the Maclaurin series for $(1-x)^{-1}$ is the geometric
series $1 + x^2 + x^3 + \cdots$. So then we have that
\[ G = x^k(1-x)^{-k} \]
We can then use the theorem found in lecture 23 to rewrite this as a sum,
as follows:
\[ G = x^k \sum_{n \geq 0} \binom{k+n-1}{n} x^n \]
\[ G = \sum_{n \geq 0} \binom{k+n-1}{n} x^{n+k} \]
Then we can use the fact that $\binom{n}{k} = \binom{n}{n-k}$ to write this
as:
\[ G = \sum_{n \geq 0} \binom{k+n-1}{k-1} x^{n+k} \]
\[ G = \sum_{n \geq k} \binom{n-1}{k-1} x^n \]
So we have written our generating function in the desired form.
\end{document}

%%%%%%%%%%%%%%%%%%%%%%%%%%%%%% Preamble
\documentclass{article}
\usepackage{amsmath,amssymb,amsthm,fullpage}
\usepackage[a4paper,bindingoffset=0in,left=1in,right=1in,top=1in,
bottom=1in,footskip=0in]{geometry}
\newtheorem*{prop}{Proposition}
%\newcounter{Examplecount}
%\setcounter{Examplecount}{0}
\newenvironment{discussion}{\noindent Discussion.}{}
\setlength{\headheight}{12pt}
\setlength{\headsep}{10pt}
\usepackage{fancyhdr}
\pagestyle{fancy}
\fancyhf{}
\lhead{Ma6a Pset 5}
\rhead{Matt Lim}
\pagenumbering{gobble}
\pagenumbering{gobble}
\begin{document}

%%%%%%%%%%%%%%%%%%%%%%%%%%%%%% Problem 1
\section*{Problem 1}
Given an edge $e = (a_1, a_2)$, where $a_1 \in V_1$, $a_2 \in V_2$,
here is an algorithm to generate a perfect matching of $G$
that contains $e$. So, we will keep track of $S_1$, $S_2$, and $M$, where $S_1$
contains the vertices in the matching so far from $V_1$, $S_2$ contains the
vertices in the matching so far from $V_2$, and $M$ contains the edges in the
matching so far. So to start, $M$ will just contain $e$, and $S_1$ and $S_2$
will contain the respective vertices. Then we will iterate
through the vertices in $V_1$. For each vertex $v$, if it has an edge to a
vertex not already in $S_2$, then we can just add that edge and its respective
vertices to $M$, $S_1$, and $S_2$. If all of $v$s edges are to vertices in $S_2$,
then we can do the following. We have that there must be edges from vertices in
$S_1$ to vertices not in $S_2$; otherwise, we would clearly have that
$|A| \geq |N(A)|$ for the subset $A = S_1$. So we can just add one of the
vertices $u$ adjacent to one of the vertices in $S_1$ to $S_2$ (such that $u$
is not already in $S_2$), add $v$ to $S_1$, and add/switch the edges around
in $M$ accordingly. That is, we basically switch a vertex in $S_1$ to match
with something else so that $v$ can match with a vertex in $S_2$.

%%%%%%%%%%%%%%%%%%%%%%%%%%%%%% Problem 2
\section*{Problem 2}
We will format this problem as a graph problem. Let one set of vertices $V_1$
be those vertices that represent each distinct type of card, and let one set
of vertices $V_2$ be those vertices that represent each column in the
$n \times m$ matrix. So we have that $|V_1| = |V_2| = m$. Let $G$ be the
connected undirected bipartite graph $G = (V_1 \cup V_2, E)$, where $E$ are the
edges connecting $V_1$ and $V_2$. We have that for each vertex $v \in V_1$,
there are $n$ edges going from it to vertices in $V_2$. This is because each
distinct type of card appears $n$ times, so it can appear in at most $n$
columns. Then we have that for each vertex $v \in V_2$, there are $n$ edges
going into it from vertices in $V_1$. This is because each column has $n$ rows,
so each column can have $n$ cards. Then we have that by
the variant of Hall's Theorem shown in lecture 12,
we have a perfect matching (a matching of size $|V_1|$), since for every subset
$A$ of $V_1$ we have that $|A| \leq |N(A)|$. This fact is clear given that
each vertex in $V_1$ has $n$ outgoing edges and each vertex in $V_2$ has
$n$ incoming edges. Therefore, we can conclude
that there is always a matching for $G$ that matches every vertex in $V_1$ with
a vertex in $V_2$. Notice that this the same as saying that every column can be
matched with a distinct type of card.
But due to our formulation, this means that there exists a set of $m$ cards such
that each is of a different type and each is in a different column. To elaborate,
our graph represents all the ways that the $m$ distinct types of cards can
fall into the $m$ columns. And since we always have a perfect matching in our
graph (which means each column can be matched with a distinct type),
the desired statement is true. Thus we have proved the desired statement.

%%%%%%%%%%%%%%%%%%%%%%%%%%%%%% Problem 3
\section*{Problem 3}
Here is the strategy:
\begin{enumerate}
    \item Biscuit finds a maximum matching $M$.
    \item Biscuit starts by choosing a vertex $v_1$ not in $M$.
    \item Biscuit, when it is his turn, always chooses an edge that is in $M$.
    \item Biscuit wins.
\end{enumerate}
Before proving our strategy works we will first prove a lemma. The lemma
is that if we have a maximum matching $M$, then there does not exist
an path in $M$ that starts and ends on unmatched vertices. So, consider
an arbitrary matching $M$, and assume to the contrary that there is such
a path $p$ that starts and ends on unmatched vertices. Then we can increase
the cardinality of our matching by first taking all the edges in $p$ out
of our matching, then adding back in alternating edges along $p$, starting
with the first edge of $p$. We have that this keeps $M$ a matching because
we alternate in $p$, and the start and end vertices of $p$ were unmatched
to begin with, so making those matched is fine. And we have that this
increases the cardinality of $M$ because before in $p$, the maximum number
of edges we could match was by alternating edges, not including the starting
and ending edge (because the starting and ending vertices were unmatched).
So now that we alternate along the whole path, we can clearly match more, since
before we were alternating along a path shorter by two edges.

Now we will prove that our strategy works using our lemma. So, we have that
Biscuit starts by choosing a vertex $v_1$ not in $M$. Then,


%%%%%%%%%%%%%%%%%%%%%%%%%%%%%% Problem 4
\section*{Problem 4}
We will prove this to be true. So, assume to the contrary that if the size
of the maximum flow is not a multiple of five, then not every maximum flow
of the network has a non-zero flow through $e$. That is, there exists some
maximum flow that has zero flow through $e$. Let us call this flow $f$.
We have that since the size
of the maximum flow is not a multiple of five, the size of the min cut is not
a multiple of five (by the max flow min cut theorem). This clearly means that
the min cut goes through $e$, since all the other edges have capacities that
are multiples of five. Then we have
that $e$ must be saturated in $f$, since the cut goes through $e$ (we proved this
in lecture 13, slide 14). But this is a contradiction, since we said that $f$
had zero flow through $e$. Thus we have proved our statement.

%%%%%%%%%%%%%%%%%%%%%%%%%%%%%% Problem 5
\section*{Problem 5}
Here is our algorithm.
\begin{enumerate}
    \item Give all edges in the graph $G = (V, E)$ a capacity of $\infty$.
    \item Make all vertices, except for $s$ and $t$,
            into edges with capacity $1$. So if we have
        a vertex $v$, it will become an edge $e = (u,v)$, where the capacity
        of $e$ is $1$. Call this modified graph $G' = (V', E')$, where
        $E' = E + X$, where $X$ are the edges spawned by the vertices.
    \item Find the edges in the min cut of $G'$ and return the vertices that
        spawned those edges. To find the min cut, we can first find the
        capacity of the min cut using Ford-Fulkerson (covered in class). We
        can then use BFS to find a cut whose capacity is equivalent to the
        capacity of the min cut (this was also covered in class).
\end{enumerate}
Now we will prove that this algorithm is correct. We have that the min cut
disconnects the graph by definition. We also have that it disconnects the
graph by minimizing the capacity of the edges it cuts. This means that
the edges we cut will have to be in $X$, since the edges in $E$ have infinite
capacity. Then we have that since we assigned all the edges in $X$ to have
a capacity of $1$, and the edges in $X$ actually represent vertices in $G$,
finding the min cut in $G'$ is the same as finding the minimum set of
disconnecting vertices in $G$. Thus our proof is complete.

\end{document}

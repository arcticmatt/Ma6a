%%%%%%%%%%%%%%%%%%%%%%%%%%%%% Preamble
\documentclass{article}
\usepackage{amsmath,amssymb,amsthm,fullpage}
\usepackage[a4paper,bindingoffset=0in,left=1in,right=1in,top=1in,
bottom=1in,footskip=0in]{geometry}
\usepackage{algorithm}
\newtheorem*{prop}{Proposition}
%\newcounter{Examplecount}
%\setcounter{Examplecount}{0}
\newenvironment{discussion}{\noindent Discussion.}{}
\setlength{\headheight}{12pt}
\setlength{\headsep}{10pt}
\usepackage{fancyhdr}
\pagestyle{fancy}
\fancyhf{}
\lhead{Ma6a Pset 3}
\rhead{Matt Lim}
\pagenumbering{gobble}
\pagenumbering{gobble}
\begin{document}

%%%%%%%%%%%%%%%%%%%%%%%%%%%%%% Problem 1
\section*{Note}
I will use "lower level" to mean a level that is higher up in the BFS tree
(root node is level 0), and "higher level" to mean a level that is lower down in
the BFS tree (root node is level 0). In other words, "lower" and "higher"
refer to the numerical number we assign to a level, where the root
node is assigned level 0 and so on.
\section*{Problem 1}
\begin{description}
    \item[(a)] The minimum possible value that $\alpha(G,s)$ can have is
        $|V| - 1$. The graphs that yield this are those graphs that just have
        one root/source node with only leaf nodes
        (nodes with no child nodes) and complete graphs. For the former,
        for the $\alpha$ value to be minimized, the BFS must be done from the
        node I specified as root/source.
        The maximum possible value that $\alpha(G,s)$ can have is
        $\sum_{i=1}^{|V|-1} i = \frac{|V|(|V|-1)}{2}$.
        The graphs that yield this are those graphs
        that have just one node at each level (so basically graphs that
        just look like lines), and the source node would be a node at the
        very end.
    \item[(b)] See the attached page for the counterexample disproving this.
\end{description}

%%%%%%%%%%%%%%%%%%%%%%%%%%%%%% Problem 2
\section*{Problem 2}
\begin{description}
    \item[(a)] A graph edge that is not in the BFS tree:
        \begin{itemize}
            \item Can be between vertices of the same level
            \item Can be between vertices in consecutive levels (from
                higher levels to lower levels and from lower levels to higher
                levels)
            \item Can be edges from higher levels to lower levels (these
                do not have to be between vertices in consecutive levels)
            \item Can be those edges going outwards from vertices that are not
                connected to the source node
        \end{itemize}
        The type that should appear in the shortest paths graph is the second
        type (from lower levels to higher levels, as the opposite is going
        backwards), as edges between vertices of the same level add uneccessary
        distance between vertices, and edges of the third type are directed
        back towards lower levels (going backwards). Edges of the fourth type
        obviously should not appear in the shortest paths graph since they
        are not even connected to the source node.
        The fact that an edge $e$
        between vertices $u,v$ of respective levels $i-1,i$ is on a shortest
        path was proved in lecture 7.
    \item[(b)] NOTE: IGNORE THE G' NOTATION IN THE PROBLEM. WE WILL USE
        OUR OWN NOTATION.
        The algorithm is as follows
        \begin{enumerate}
            \item We run BFS on our original graph $G$ from $s$ to obtain
                a BFS tree $G'$
            \item We alter $G'$ by adding back in those edges that are between
                consecutive levels and that go from lower levels to higher
                levels. These edges need to be in $G$.
            \item We now make a new BFS tree $G''$ by running a BFS on a
                graph $G_R$ from $t$, where $G_R$ is the same as $G$ but with
                all the edges reversed.
            \item We alter $G''$ by adding in those edges that are between
                consecutive levels and that go from lower levels to higher
                levels. These edges need to be in $G_R$.
            \item We then reverse all the edges of $G''$.
            \item Now we have that the shortest paths graph $G_S = G' \cap G''$.
        \end{enumerate}
        We can see that this works for the following reason. $G'$, after
        step 2, has only the edges (and vertices) that could possibly be in a
        shortest path from $s$ to $t$ (edges between vertices in consecutive
        levels, going from lower levels to high levels). That is, it contains
        all the shortest paths from $s$ to $t$. But it also contains some extra
        edges and vertices. Then, intersecting
        this with $G''$ gets rid of those edges (and vertices) that are not
        actually on a path to $t$. This is because those edges and vertices
        will not be included in the BFS tree of $G_R$.
        Thus we are just left with our shortest paths graph.
\end{description}

%%%%%%%%%%%%%%%%%%%%%%%%%%%%%% Problem 3
\section*{Problem 3}
\begin{description}
    \item[(a)] See the attached page for the counterexample disproving this.
    \item[(b)] See the attached page for the counterexample disproving this.
\end{description}

%%%%%%%%%%%%%%%%%%%%%%%%%%%%%% Problem 4
\section*{Problem 4}
The algorithm is as follows
\begin{enumerate}
    \item We make a graph with 6 vertices. Each vertex represents a number
        from 1-6, and no two vertices can represent the same number. So we have
        6 vertices, where one vertex represents 1, one vertex represents 2,
        etc.
    \item For each domino with numbers $x$ and $y$, we draw a line between
        the vertices that represent $x$ and $y$.
    \item Then we have that if there is a Eulerian path through this graph,
        Biscuit can place all the dominos in one row. And we learned in lecture
        that a connected undirected graph contains a Eulerian path if and only
        if it contains exactly two vertices with odd degrees. So we can just
        check for this to see if there is a Eulerian path, and if there is,
        we return true, else, we return false.
\end{enumerate}
We can see that this works for the following reason. An Eulerian path in a
graph $G$ is a path that passes through every edge of $G$ exactly once.
And in our graph, passing through an edge is equivalent to using a domino,
and passing through consecutive edges is equivalent to placing dominos
next to each other (within the rules). Thus it follows that a Eulerian path
through $G$ means that we can place all the dominos next to each other, within
the rules, using each domino exactly once.
%%%%%%%%%%%%%%%%%%%%%%%%%%%%%% Problem 5
\section*{Problem 5}
See the attached page for the counterexample disproving this.

\end{document}

%%%%%%%%%%%%%%%%%%%%%%%%%%%%%% Preamble
\documentclass{article}
\usepackage{amsmath,amssymb,amsthm,fullpage}
\usepackage[a4paper,bindingoffset=0in,left=1in,right=1in,top=1in,
bottom=1in,footskip=0in]{geometry}
\usepackage[utf8]{inputenc}
\usepackage[T1]{fontenc}
\usepackage{textcomp}
\usepackage{gensymb}
\newtheorem*{prop}{Proposition}
%\newcounter{Examplecount}
%\setcounter{Examplecount}{0}
\newenvironment{discussion}{\noindent Discussion.}{}
\setlength{\headheight}{12pt}
\setlength{\headsep}{10pt}
\usepackage{fancyhdr}
\usepackage{pbox}
\pagestyle{fancy}
\fancyhf{}
\lhead{Ma6a Pset 8}
\rhead{Matt Lim}
\pagenumbering{gobble}
\pagenumbering{gobble}
\begin{document}

%%%%%%%%%%%%%%%%%%%%%%%%%%%%%% Problem 1
\section*{Problem 1}
We want to ennumerate all the symmetries of the square and the number of
elements that they fix, where the elements are the possible colorings
of $C$. Given the problem, we have that there are clearly $4^6$ such elements,
since each of $6$ sides can be colored with one of $4$ colors.

\vspace{3mm}

\begin{tabular}{ | l | c | c | }
    \hline
    Symmetry $g$ & Num Symmetries & $F(g)$ \\ \hline
    Identity & 1 & $4^6$ \\ \hline
    \pbox{40cm}{Rotation around an axis from the \\center of one face to the center
    of the opposite face by $90 \degree$} & 6 (3 axes, 2 per axis) &
    $6 \cdot 4^3$ \\ \hline
    \pbox{40cm}{Rotation around an axis from the \\center of one face to the center
    of the opposite face by $180 \degree$} & 3 (3 axes, 1 per axis) &
    $3 \cdot 4^4$ \\ \hline
    \pbox{40cm}{Rotation around an axis from the \\center of an edge to the center
    of the opposite edge by $180 \degree$} & 6 (6 axes, 1 per axis) &
    $6 \cdot 4^3$ \\ \hline
    \pbox{40cm}{Rotation around a body diagonal of the cube by\\
    by $120 \degree$} & 8 (4 axes, 2 per axis) &
    $8 \cdot 4^2$ \\ \hline
\end{tabular}

\vspace{3mm}

Now we will explain this table in more depth. Explaining the identity element
is easy. This rotation doesn't affect the cube at all, so it fixes all
$4^6$ possible colorings.

Now we will explain the symmetry where the cube is rotated around an axis from
the center of one face to the center of the opposite face by $90 \degree$.
We have that in this rotation, two faces are unaffected (the faces which the
axis passes through). Since these faces are unaffected by the rotation,
we can choose any color for either of them since they will always be fixed.
The other four faces all rotate. This means that all four other faces must be
the same color - otherwise, the rotation will not fix the coloring.
So, we can choose any color for one face, any color for another face,
and any color for the other four faces. This gives us $4^3$ colorings per
rotation.

Now we will explain the symmetry where the cube is rotated around an axis from
the center of one face to the center of the opposite face by $180 \degree$.
We have that in this rotation, two faces are unaffected (the faces which the
axis passes through). Since these faces are unaffected by the rotation,
we can choose any color for either of them since they will always be fixed.
The other fouce faces all rotate. But this time, they rotate $180 \degree$.
So we have that if parallel faces are the same color, the rotation will still
fix the coloring. This gives us more options than the last symmetry. Specifically,
we now have that two of the remaining four faces must be colored with the same
color, and the other two of the remaining four faces must also be colored with
the same color. So, we can choose any color for one face, any color for another
face, any color for two faces, and any color for another two faces. So this gives
us $4^4$ colorings per rotation.

Now we will explain the symmetry where the cube is rotated around an axis from
the center of an edge to the center of the opposite edge by $180 \degree$.
This rotation is harder to visualize, but it essentially does three swaps,
and thus creates three groups. To see this, picture a dice. Let the front face
be two, the right face be three, the left face be four, the back face be five,
the top face be one, and the bottom face be six. Now, WLOG, consider the rotation
around the axis that goes from the midpoint of the bottom of the front face
(which has two) to the midpoint of the back of the top face (which has one). We can
see that this rotation causes the front face to be six, the right face to
be four, the left face to be three, the back face to be one, the top face to be
five, and the bottom face to be two. Now we can see that we have three groups:
the left and right faces, the top and back faces, and the bottom and front faces.
The faces in these groups swap when this kind of rotation is applied,
and thus we can see that the colorings that are fixed are the colorings that
color each group one color. Another way to see this is that
$1 \rightarrow 5, 3 \rightarrow 4, 3 \rightarrow 3, 5 \rightarrow 1, 2 \rightarrow
6, 6 \rightarrow 2$ which means we have cycles of $(15),(34),(26)$ which must
each be colored with the same color.
Then, since we have three cycles, this gives us $4^3$ colorings per rotation.

Finally, we will explain the symmetry where the cube is rotated around a body
diagonal of the cube by $120 \degree$. This rotation is harder to visualize,
so we will describe it as follows. Picture a dice. Let the front face
be two, the right face be three, the left face be four, the back face be five,
the top face be one, and the bottom face be six. Now, WLOG, consider the
positive rotation around the axis that goes from the bottom-front-left corner
to the top-back-right corner. This has the following effect:
\[ 1 \rightarrow 3 \]
\[ 2 \rightarrow 6 \]
\[ 3 \rightarrow 5 \]
\[ 4 \rightarrow 2 \]
\[ 5 \rightarrow 1 \]
\[ 6 \rightarrow 4 \]
So we can see that there are two cycles,
\[ (135) \text{ and } (264) \]
Then we have that each cycle must be colored the same color. That is,
sides $1,3,5$ must be the same color and sides $2,6,4$ must be colored the
same color. This is because the rotation causes these sides to cycle with
each other, and so in order to have the rotation fix the coloring, all
the colors in the cycle must be the same. So since we have two cycles,
this gives us $4^2$ colorings per rotation.

So we get that there are $24$ symmetries in total and that they fix a total of
$5760$ elements. Then we have that
\[ \frac{1}{|G|} \sum_{g \in G} F(g) = \frac{5760}{24} = 240 \]
But this is just the equation for the number of distinct orbits under the
symmetries we defined, which is equivalent to the number of distinct colorings
under rotations of the cube. So we have that the number of distinct colorings
is $240$.

%%%%%%%%%%%%%%%%%%%%%%%%%%%%%% Problem 2
\section*{Problem 2}
Assume to the contrary that there does not exist a permutation $g \in G$ that
does not contain any cycles of length one in its cycle structure. This means
that every permutation $g \in G$ contains a cycle of length one in its cycle
structure. Let us consider $t$, the number of orbits for this setup. We have
that
\[ t = \frac{ \sum_{g \in G} F(g) }{|G|} \]
Let $F(g)$ be defined as in lecture 21 (the number of stabilizers that contain
$g$). We have that $F(id) = |X|$, because $id$ is in every stabilizer. Then,
for every other $g \in G$, we have that $F(g) \geq 1$. This is because
every permutation $g \in G$ contains a cycle of length one, so every permutation
$g$ is in at least one stabilizer. Then we can write
\[ t \geq \frac{ |X| + |G| - 1 }{|G|} \]
Then we have that $|X| \geq |G| + 2$. So this gives us
\[ t \geq \frac{ 2|G| + 1 }{|G|} \]
\[ t > 2 \]
or that the number of orbits is greater than two. But this is a contradiction.
Thus we can conclude that there exists a permutation $g \in G$ that does not
contain any cycles of length one in its cycle structure.

%%%%%%%%%%%%%%%%%%%%%%%%%%%%%% Problem 3
\section*{Problem 3}
First we will prove that $N \cap H$ is a group (with the same binary operation
$*$ as $G$). First, we have that for every element $x \in N \cap H$, there exists
$x^{-1} \in N \cap H$ such that $x * x^{-1} = x^{-1} * x = e$. We have that this
is true because $x \in N$ and $x \in H$, which means that $x^{-1} \in N$
and $x^{-1} \in H$ (because $N$ and $H$ are subgroups) which then implies that
$x^{-1} \in N \cap H$ as desired. Second, we have that for every $x,y \in
N \cap H$, we have $x * y \in N \cap H$. We have that this is true because
$x, y \in N \implies x * y \in N$ and $x, y \in H \implies x * y \in H$ (because
$N$ and $H$ are subgroups). So we have $x * y \in N \cap H$ as desired. Third,
we have that there exists $e \in N \cap H$ such that for every $x \in N \cap H$,
we have $e * x = x * e = x$. This is true because if $e$ is the identity of $G$,
then $e \in N$ and $e \in H$ because these are subgroups. Thus, $e \in N \cap H$
and we have our identity element as desired. Fourth and last, we have that
associativity is satisfied by the associativity of $G$ (since elements in
$N \cap H$ are a subset of elements in $G$). So, since we have proved
closure, associativity, identity, and inverse for $N \cap H$ under $*$, we have
that it is a group. Then we have that $N \cap H$ is a subgroup of $N$ since
its set of members is a subset of $N$ and that $N \cap H$ is a subgroup of
$H$ since its set of members is a subset of $H$. Let $m = |N \cap H|$, $n = |N|$,
and $k = |H|$. Then, by Lagrange's theorem, we have that $m | n$ and that
$m | k$. But $n$ and $k$ are relatively prime. Thus, we have that $m = 1$, which
means that $N \cap H$ is the group containing only the identity.

Now consider, for any $x \in H$ and $y \in N$,
\[ x * y * x^{-1} * y^{-1} \]
Recall the definition of a normal subgroup $N$ of $G$ that
$\forall n \in N$, $\forall g \in G$, $gng^{-1} \in N$. We can see that this
definition can be easily derived from the given definition, because
if $gNg^{-1} = N$ for every $g \in G$, then $\forall n \in N$,
$gng^{-1} \in N$ for every $g \in G$ must be true. If it were not, then
$gNg^{-1} = N$ would not hold for every $g \in G$.
We have that $x * y * x^{-1} \in N$ because $N$ is a normal subgroup,
$x \in G$ (because $x \in H$, and $H$ is a subgroup of $G$), and $y \in N$.
We also have that $y^{-1} \in N$ because $N$ is a
group. Then we have that $y * x^{-1} * y^{-1} \in H$ because $H$ is a normal
subgroup, $y \in G$ (because $y \in N$ and $N$ is a subgroup of $G$),
and $x^{-1} \in H$ (because $H$ is a group). We also
have that $x \in H$ (we defined this). Then we have that
\[ x * y * x^{-1} * y^{-1} \in N \cap H \]
This is because $x * y * x^{-1} * y^{-1}$ is the result of operating on
two elements in $N$, and thus is in $N$, and is also a result of operating on
two elements of $H$, and thus is in $H$.
But then, given what $N \cap H$ contains, we have:
\[ x * y * x^{-1} * y^{-1} = e \]
\[ x * y * x^{-1} = e * y = y \]
\[ x * y = y * x \]
as desired.


%%%%%%%%%%%%%%%%%%%%%%%%%%%%%% Problem 4
\section*{Problem 4}
To do this problem, we will consider the $\log_2$ transform of the recurrence
and transform our answer back at the end. So, our recurrence becomes
\[ a_{i+2} = \frac{a_{i+1}}{2} + \frac{a_i}{2} \]
Then we can write out a transformed $A(x)$ as follows, taking $\log_2$ of
$a_0$ and $a_1$ to keep everything transformed.
\[ A(x) = a_0 + a_1x + (\frac{a_1}{2} + \frac{a_0}{2})x^2 + (\frac{a_2}{2}
    + \frac{a_1}{2})x^3 + (\frac{a_3}{2} + \frac{a_2}{2})x^4 + \cdots \]
\[ A(x) = a_0 + a_1x + \frac{x}{2}(A(x) - a_0) + \frac{x^2}{2}A(x) \]
\[ A(x) = 1 + 3x + \frac{x}{2}(A(x) - 1) + \frac{x^2}{2}A(x) \]
\[ -\frac{x^2}{2}A(x) - \frac{x}{2}A(x) + A(x) = 1 + 3x - \frac{x}{2} \]
\[ -x^2A(x) - xA(x) + 2A(x) = 2 + 6x - x \]
\[ A(x) = \frac{2 + 5x}{-x^2 - x + 2} \]
Use partial fraction decomposition (skip writing b/c uninteresting technical
steps) to get...
\[ A(x) = -\frac{8}{3} \cdot \frac{1}{x+2} + \frac{7}{3} \cdot \frac{1}{1-x} \]
\[ A(x) = -\frac{8}{6} \cdot \frac{1}{1 - (-\frac{x}{2})}
    + \frac{7}{3} \cdot \frac{1}{1-x} \]
Use definition given on slide 5, lecture 23 to get...
\[ A(x) = -\frac{4}{3} \cdot \sum_{n \geq 0} (-\frac{1}{2})^nx^n
    + \frac{7}{3} \cdot \sum_{n \geq 0} 1^nx^n \]
\[ A(x) = \sum_{n \geq 0} \left( (-\frac{4}{3})(-\frac{1}{2})^n + (\frac{7}{3})
    (1^n)\right) x^n \]
We can see that as $n \rightarrow \infty$, $(-\frac{4}{3})(-\frac{1}{2})^n$
goes to $0$ because $(-\frac{1}{2})^n$ gets really small. So then we are left
with just $\frac{7}{3} \cdot 1^n$ as our coefficient $a_n$, which just goes to
$\frac{7}{3}$ as $n \rightarrow \infty$. So, in our transformed version, we
have that $\lim_{n \rightarrow \infty} a_n = \frac{7}{3}$.
Now we just need to untransform this answer to get our final answer. Since we
originally transformed our problem by taking $\log_2$ of everything, we can
just do the following to get our final answer (raise our transformed answer
to the power of two):
\[ \lim_{n \rightarrow \infty} a_n = 2^{\frac{7}{3}} \]

\end{document}

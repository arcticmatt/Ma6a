%%%%%%%%%%%%%%%%%%%%%%%%%%%%%% Preamble
\documentclass{article}
\usepackage{amsmath,amssymb,amsthm,fullpage}
\usepackage[a4paper,bindingoffset=0in,left=1in,right=1in,top=1in,
bottom=1in,footskip=0in]{geometry}
\newtheorem*{prop}{Proposition}
%\newcounter{Examplecount}
%\setcounter{Examplecount}{0}
\newenvironment{discussion}{\noindent Discussion.}{}
\setlength{\headheight}{12pt}
\setlength{\headsep}{10pt}
\usepackage{fancyhdr}
\pagestyle{fancy}
\fancyhf{}
\lhead{Ma6a Pset 4}
\rhead{Matt Lim}
\pagenumbering{gobble}
\pagenumbering{gobble}
\begin{document}

%%%%%%%%%%%%%%%%%%%%%%%%%%%%%% Problem 1
\section*{Problem 1}
Here we have two cases.

The first is if $d_i \geq i - 1$ for all $i$. Then
we have that $k = 1 + \max_i \min\{d_i, i-1\} = 1 + n - 1 = n$. In this
case, we can just cover each vertex a unique color. So clearly the graph is
$k = n$ colorable.

The second case is if we have $d_i \geq i - 1$ for some $i$
and $ d_i < i - 1$ for other $i$. In this case, there is a crossover point
$i = x$ such that when $i = x - 1$, $d_{x-1} \geq x - 2$ and when $i = x$,
$d_{x} < x - 1$. Notice that here,
\[ i < x - 1 \implies \min\{d_i, i - 1\} = i - 1 \]
\[ i > x \implies \min\{d_i, i - 1\} = d_i \]
Given the nature of $i$ and $d_i$ (one increases monotonically as the other
decreases monotonically), we then have that
\[ \text{for } i < x - 1 \text{, } \max_i \min\{d_i, i-1\} = x - 3 \]
\[ \text{for } i > x \text{, } \max_i \min\{d_i, i-1\} = d_{x+1} \]
Then we have that
\[ \text{for } i = x - 1 \text{, } \max_i \min\{d_i, i-1\} = x - 2 \]
\[ \text{for } i = x \text{, } \max_i \min\{d_i, i-1\} = d_x \]
We also have that
\[ d_x < x - 1 \implies d_x \leq x - 2 \]
So now we know the max of $\min\{d_i, i-1\}$ for all $i$ is $x - 2$.
So overall, we can conclude that
\[ k = 1 + \max_i \min \{d_i, i - 1\} = 1 + x - 2 = x - 1 \]
Now we must show that our graph is $k = x - 1$ colorable. To do this, we can
give a method to color $G$. We can start by coloring $\{v_1, \cdots, v_{x-1}\}$
each a different color. Then, for $v_i$ where $i \geq x$, we have that
$d_i < x - 1$. So we have that we can color it a color such that it is not
connected to any vertices of that same color, because at most $v_i$ is connected
to $x - 2$ colors (so just color it the remaining color).

So we have covered both cases and proved the desired statement ($G$ is
$k$-colorable) for both.

%%%%%%%%%%%%%%%%%%%%%%%%%%%%%% Problem 2
\section*{Problem 2}
We can see that adding an edge $e = (u,v)$ to a MST will always create
at least one circle. This is because in any spanning tree there is exactly one
path between any two vertices $u$ and $v$. So then we have that adding $e = (u,v)$
will create a circle that includes the path from $u$ to $v$ in the spanning
tree and the new edge $e$.

Now we must show that adding an edge $e = (u,v)$ to a MST will not create
more than one circle. This is simple. We have that the only situation in which
one can create more than one simple cycle in a graph by adding one edge is if there
is already a cycle in the graph. But we have that spanning trees do not have
any cycles. So we cannot possibly create more than one simple cycle by
adding one edge to an MST. So our proof is complete.

%%%%%%%%%%%%%%%%%%%%%%%%%%%%%% Problem 3
\section*{Problem 3}
Here is a description of our algorithm:
\begin{enumerate}
    \item Run regular Prim's on the graph $G$ to get the cost of a MST
        for $G$. Let that cost be called $C_{MST}$.
    \item Let $E'$ be the edges adjacent to the vertices $V'$.
    \item Let $G' = (V-V', E-E')$. If $G'$ is not connected, return false,
        as at least one vertex $v \in V'$ cannot be a leaf in a
        spanning tree for $G$.
    \item Else, if $G'$ is connected, then run Prim's on $G' = (V-V', E-E')$
        to get a MST $M'$.
    \item Let $E''$ be the subset of $E'$ such that each $e \in E''$ is adjacent
        to a vertex in $V$ and a vertex in $V'$. That is, no edge $e \in E''$
        connects two vertices in $V'$. Then add $V'$ and $E''$ back to $G'$. Call
        this new graph $G''$. Then, for each vertex $v \in V'$, add the edge
        $e_m$ that connects it to $M'$ with the least weight added. If there
        is no such edge that connects $v$ to $M'$, return false. Let the
        the new spanning tree be called $M$.
    \item Compare the cost of $M$ with $C_{MST}$. If they are equal,
        $M$ is our desired MST. Else, it is impossible to find a MST of the type
        we want, so return false.
\end{enumerate}
Now we will explain why our algorithm is correct. First we will explain step
3. We have that if we remove $V'$ and the graph $G'$ is no longer connected,
we return false. This because this implies that in any spanning tree, one
$v \in V'$ must have at least two edges adjacent to it. If this were not
true, then the spanning tree would not be connected (and thus would not
actually be a spanning tree). Then we have that we get a spanning tree $M'$
for $G'$, then connect the vertices in $V'$ back to this $M'$ via edges
with the least weight (step 5). This gives us a spanning tree because we clearly
span all the vertices, and we don't create any cycles because we are not
connecting any of the vertices in $V'$ with other vertices in $V'$. We also
have that it enforces that all the vertices in $V'$ are leaves, since
for each vertex $v \in V'$ we just add one edge that connects it to $M'$. If
we cannot find an edge that connects it to $M'$, we return false, because
this means that not all vertices in $V'$ can be leaves.
Then we check if the spanning tree generated by connecting the
vertices in $V'$, $M$, is actually a \textit{minimum} spanning tree. Now we must
explain why this algorithm will always find such a MST if it does exist. This is
fairly simple; we have that any MST for any graph $G = (V, E)$
such that all vertices in $V' \subset V$ are leaves
will be of the same form as the spanning trees we generate with our algorithm.
That is, each $v \in V'$ will be connected to one vertex $v'$ in some MST for
the rest of $G$, since the MST will cover the rest of the vertices by
definition. And it doesn't matter if there are multiple MSTs for the rest of
the vertices, as they will all span all those vertices and will all have the
same cost. So it doesn't matter which MST $M'$ we find in our algorithm.
Another way to explain why any MST for any graph $G$ such that
all vertices $V' \subset V$ are leaves will be of the same form as the spanning
trees we generate with our algorithm is the following. Such MSTs are going to have
to have all vertices $V - V'$ connected minimimally, which is why we run
Prim's on those vertices in our algorithm. And then all vertices in $V'$ need
to be leaves and need to be connected with minimum edge weights as well, which
is exactly what we do in our algorithm. Thus we can conclude that our algorithm
is correct.

%%%%%%%%%%%%%%%%%%%%%%%%%%%%%% Problem 4
\section*{Problem 4}
We will prove this is true by induction. We will induct on $k$, the lower
bound of the edge weights we remove. Our base case will be the lowest edge
weight in the graph. We can see that in this case, we will remove every
edge from $T_1$ and $T_2$. So clearly, $T_1'$ and $T_2'$ have the same
connected components, since both graphs are exactly the same. Now comes our
inductive assumption. So, assume that for $k = n$, $T_1'$ and $T_2'$ have
the same connected components. Now we must show that for $k = n+1$,
$T_1''$ and $T_2''$ have the same connected components, where $T_1''$ and
$T_2''$ are the trees that result from adding back edges of weight
$n \leq w < n+1$ from $T_1$ or $T_2$ to $T_1'$ and $T_2'$, respectively (in other
words, $T_1''$ and $T_2''$ are the graphs that result from throwing from
$T_1$ and $T_2$ every edge weight of at least $n+1$). To show this, we will
consider three cases. The first case is if going from $k=n$ to $k=n+1$ results
in no edges being added. That is, $T_1'' = T_1'$ and $T_2'' = T_2'$. This case
is trivial; before we had the same connected components, so afterwards we do
as well. The second case is if going from $k=n$ to $k=n+1$ results in the
same set of edges being added to $T_1'$ and $T_2'$. In this case, it is
also clear that $T_1''$ and $T_2''$ have the same connected components, since
we added the same set of edges to each. The third case is if going from $k=n$
to $k=n+1$ results in a different set of edges being added to $T_1'$ and
$T_2'$ where those edges end up connecting the same connected components. That is,
we add a set of edges $S_1$ to $T_1'$, and we add a set of edges $S_2$ to
$T_2'$, where $S_1$ can differ from $S_2$, but both sets of edges end up
connecting the same connected components. Since we connect the same connected
components, then clearly in this case $T_1''$ and $T_2''$ have the same connected
components. The fourth case is if going from $k=n$
to $k=n+1$ results in $T_1''$ and $T_2''$ having different connected components.
But we have that this is not actually a case, because it cannot actually happen.
To see this, assume to the contrary that $T_1''$ and $T_2''$ have different
connected components. Then, WLOG, we have that there are connected components
$X$ and $Y$ of $T_1'$ that are connected in $T_1''$ that are not connected in
$T_2''$. But we are going to have to connect those components in $T_2$
eventually, and when we do so, we are going to use an edge of weight at least
$n+1$ (since we will add it for a greater $k$). But this is a contradiction,
since this means that $T_2$ is not a MST (since $X$ and $Y$ were connected
with a weight of $n \leq w < n+1$ in $T_1$). Thus we can conclude by
induction that $T_1'$ and $T_2'$ have the same connected components for all
$k$.

%%%%%%%%%%%%%%%%%%%%%%%%%%%%%% Problem 5
\section*{Problem 5}
\begin{description}
    \item[(a)] We will disprove this statement in the following way. We have
        that perfect matchings have $|V|/2$ edges and that spanning trees have
        $|V| - 1$ edges. So we have that any two perfect matchings for a
        spanning tree will share at least $1$ edge. Thus we can conclude that
        no spanning tree contains two edge-disjoint perfect matchings.
    \item[(b)] Let $M$ be a maximal matching, $M^*$ a maximum matching, $V$ the
        vertices in $M$, and $V^*$ the vertices in $M^*$.
        We wish to prove that $|M| \geq \frac{1}{2} |M^*|
        \implies 2 \cdot |M| \geq |M^*|$. So, assume to the contrary that $2
        \cdot |M| < |M^*|$. Notice that $2 \cdot |M| = |V|$. Also
        notice that each $v \in V$ can be adjacent to at most one edge in $M^*$,
        since $M^*$ is a matching.
        So we have that, in total, all $|V|$ vertices in $V$ are adjacent to at
        most $2 \cdot |M|$ edges in $M^*$. But since $|M^*| > 2 \cdot |M|$, this
        means there is at least one edge in $M^*$ that is not incident to any
        vertex in $M$ but is in the graph (since it is in $M^*$). So we can add
        this edge to $M$ to enlarge it.  But clearly, this is a contradiction.
        So we can conclude that $|M| \geq \frac{1}{2} |M^*|$ as desired.
\end{description}
\end{document}

%%%%%%%%%%%%%%%%%%%%%%%%%%%%%% Preamble
\documentclass{article}
\usepackage{amsmath,amssymb,amsthm,fullpage}
\usepackage[a4paper,bindingoffset=0in,left=1in,right=1in,top=1in,
bottom=1in,footskip=0in]{geometry}
\newtheorem*{prop}{Proposition}
%\newcounter{Examplecount}
%\setcounter{Examplecount}{0}
\newenvironment{discussion}{\noindent Discussion.}{}
\setlength{\headheight}{12pt}
\setlength{\headsep}{10pt}
\usepackage{fancyhdr}
\pagestyle{fancy}
\fancyhf{}
\lhead{Ma6a Pset 6}
\rhead{Matt Lim}
\pagenumbering{gobble}
\pagenumbering{gobble}
\begin{document}

%%%%%%%%%%%%%%%%%%%%%%%%%%%%%% Problem 1
\section*{Problem 1}
\begin{description}
    \item[(a)] Here is our algorithm:
        \begin{enumerate}
            \item Find the cost of the min cut of the original flow network.
                This can be done by using Ford Fulkerson to find the max
                flow, then seeing how much flow goes out of the source
                node to get the size of the max flow, then applying the max
                flow min cut theorem. Let this cost be $C_0$.
            \item Remove $e$ from the flow network and find the cost of the min
                cut for that new flow network. Let this cost be $C_1$.
            \item If $|C_1 - C_0| = c(e)$, where $c(e)$ is the capacity of $e$,
                return true. Else return false.
        \end{enumerate}
    \item[(b)] Here is our algorithm:
        \begin{enumerate}
            \item Find a cut whose size is equivalent to the size of the
                maximum flow (a min cut). This can be done by using Ford
                Fulkerson to find the max flow, then seeing how much flow
                goes out of the source node to get the size of the max flow,
                then using a BFS to find a cut whose size is equivalent
                to the size of the max flow. Then we can iterate over the
                edges and find the edges that cross the cut. Call this set of
                edges $C$.
            \item Let $count = k$. While $count > 0$ and $|C| > 0$,
                greedily remove the highest capacity edge $e_{c\_max} \in C$
                from $C$, add it to $S$, and decrement $count$ by $1$. In
                this case, since the capacities are all $1$, we can just keep
                removing any edge from $C$ and adding it to $S$ until we have
                either removed $k$ edges or there are no more edges from $C$
                to remove.
            \item Return $S$.
        \end{enumerate}
\end{description}

%%%%%%%%%%%%%%%%%%%%%%%%%%%%%% Problem 2
\section*{Problem 2}

%%%%%%%%%%%%%%%%%%%%%%%%%%%%%% Problem 3
\section*{Problem 3}

%%%%%%%%%%%%%%%%%%%%%%%%%%%%%% Problem 4
\section*{Problem 4}
\begin{description}
    \item[(a)] We will prove this is true by induction on the size of odd
        length cycles. Our base case will be cycles of length $5$. To
        prove this, consider an arbitrary cycle $(x_1x_2x_3x_4x_5)$ of
        length $5$. We have that this is just the same as the composition
        $(x_1x_2x_3)(x_3x_4x_5)$. So the base case is satisfied. Now assume
        that any odd cycle of length $k$ can be expressed as a composition of
        cycles of length $3$. Now we must show that odd cycles of length
        $k+2$ can be expressed as a composition of cycles of length $3$.
        That is, we want to show that cycles of the form $(x_1x_2\cdots x_{k+2})$
        can be expressed as a composition of cycles of length $3$. We have that
        $(x_1x_2\cdots x_{k+2})$ can be expressed as
        $(x_1x_2\cdots x_k)(x_kx_{k+1}x_{k+2})$. And due to our inductive
        assumption $(x_1x_2\cdots x_k)$ can be expressed as a combination of
        cycles of length $3$. So we can conclude that odd cycles of length
        $k+2$ can be expressed as a composition of cycles of length $3$, and
        our inductive proof is complete.
    \item[(b)]
\end{description}

%%%%%%%%%%%%%%%%%%%%%%%%%%%%%% Problem 5
\section*{Problem 5}

\end{document}

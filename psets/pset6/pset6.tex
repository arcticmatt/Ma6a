%%%%%%%%%%%%%%%%%%%%%%%%%%%%%% Preamble
\documentclass{article}
\usepackage{amsmath,amssymb,amsthm,fullpage}
\usepackage[a4paper,bindingoffset=0in,left=1in,right=1in,top=1in,
bottom=1in,footskip=0in]{geometry}
\newtheorem*{prop}{Proposition}
%\newcounter{Examplecount}
%\setcounter{Examplecount}{0}
\newenvironment{discussion}{\noindent Discussion.}{}
\setlength{\headheight}{12pt}
\setlength{\headsep}{10pt}
\usepackage{fancyhdr}
\pagestyle{fancy}
\fancyhf{}
\lhead{Ma6a Pset 6}
\rhead{Matt Lim}
\pagenumbering{gobble}
\pagenumbering{gobble}
\begin{document}

%%%%%%%%%%%%%%%%%%%%%%%%%%%%%% Problem 1
\section*{Problem 1}
\begin{description}
    \item[(a)] Here is our algorithm:
        \begin{enumerate}
            \item Find the cost of the min cut of the original flow network.
                This can be done by using Ford Fulkerson to find the max
                flow, then seeing how much flow goes out of the source
                node to get the size of the max flow, then applying the max
                flow min cut theorem. Let this cost be $C_0$.
            \item Remove $e$ from the flow network. If $e$ does not disconnect
                the networ, finish this step. Else, proceed to the next step.
                So we have that $e$ does not disconnect the network, since we
                are still in this step. Now find the cost of the min
                cut for the new flow network. Let this cost be $C_1$.
                 If $|C_0 - C_1| = c(e)$, where $c(e)$ is the capacity of $e$,
                return true. Else return false.
            \item Since we are here, we have that removing $e$ from the flow
                network disconnects it. Let $e = (a, b)$. Let $a$ be the vertex
                that is still connected to the source $s$ (or is the source)
                and let $b$ be the vertex that is still connected to the sink
                $t$ (or is the sink). Then we will connect $a$ to a new sink
                $t'$ and connect $b$ to a new source $s'$. Both edge that make
                these connections will have infinite capacity. We will then find
                the costs of the min cuts of these two new flow networks. Call
                the cost of the min cut of the flow network with $s$ and $t'$
                $C_2$ and call the cost of the min cut of the flow network
                with $s'$ and $t$ $C_3$. Then, if both $C_2 \geq C_0$ and
                $C_3 \geq C_0$, return true. Else, return false.
        \end{enumerate}
        This works for the following reasons. Step 2 works because, if removing
        $e$ does not disconnect the network, then we have that
        $|C_0 - C_1| = c(e)$ if and only if $e$ is in at least one minimum cut
        of the graph. This is true because if $e$ is in at least one minimum
        cut, then the equality must hold, since the original min cut still
        disconnects the graph, but with $c(e)$ less capacity. And if
        $|C_0 - C_1| = c(e)$, then $e$ is in at least one minimum cut because
        the new minimum cut (with cost $C_1$) must be equal to the original min cut
        without $e$. Step 3 works because, if removing $e$ disconnects the
        network, we need to find if we could have disconnected either side
        of the disconnected network with less cost. And that is exactly what
        we do. Thus our algorithm works.
    \item[(b)] Here is our algorithm:
        \begin{enumerate}
            \item Find a cut whose size is equivalent to the size of the
                maximum flow (a min cut). This can be done by using Ford
                Fulkerson to find the max flow, then seeing how much flow
                goes out of the source node to get the size of the max flow,
                then using a BFS to find a cut whose size is equivalent
                to the size of the max flow. Then we can iterate over the
                edges and find the edges that cross the cut. Call this set of
                edges $C$.
            \item Let $count = k$. While $count > 0$ and $|C| > 0$,
                greedily remove the highest capacity edge $e_{c\_max} \in C$
                from $C$, add it to $S$, and decrement $count$ by $1$. In
                this case, since the capacities are all $1$, we can just keep
                removing any edge from $C$ and adding it to $S$ until we have
                either removed $k$ edges or there are no more edges from $C$
                to remove.
            \item Return $S$.
        \end{enumerate}
        Now to briefly describe why our algorithm works. We have that every
        edge that crosses the min cut is saturated in the max flow, and that the
        edges that cross the min cut form a minimum cardinality set of edges that
        one can remove from the flow network to disconnect it (since all the
        capacities are 1). So clearly removing
        these edges is going to decrease the size of the max flow the most.
        This is because if removing one of these edges didn't decrease the max
        flow size by 1, the max flow min cut theorem would be violated for the
        new flow network (the flow network with the edge removed).
        Then we just remove as many of these edges as we can and return that
        set. In this case, it doesn't even matter which edges from the min
        cut we choose, since they all have the same capacity, and thus removing
        them affects the max flow the same.
\end{description}

%%%%%%%%%%%%%%%%%%%%%%%%%%%%%% Problem 2
\section*{Problem 2}
We want an algorithm to check whether we can remove at most $k$ edges from $G$
so that $e$ is in at least one MST of the resulting graph. We have that $G$ has
to remain connected after the removal. So, set $e = (u,v)$ and $d = w(e)$. $e$
is in at least one MST of $G$ if in every other path between $u$ and $v$ there
exists an edge with weight greater than or equal to $d$. So, we wish to remove
up to $k$ edges, so that every other path between $u$ and $v$ contains an edge
of weight greater than or equal to $d$. We can do this by removing from $G$
every $e' \in E$
with $w(e') \geq d$. Then we get a problem that was solved in lecture 14: finding
the minimum subset of edges such that its removal disconnects $u$ and $v$. Call
this minimum subset $S$. If $|S| \leq k$ and the graph $G'$ obtained by removing
$S$ from the original graph $G$ is still connected, then $S$ is our desired
set of edges. This is because by removing $S$, we have that the resulting graph
is still connected (so an MST can be formed) and that on every path from $u$
to $v$ there exists an edge with weight greater than or equal to $d$, since
we disconnected the paths that had edges all with weight less than $d$. If not,
then we cannot remove at most $k$ edges from $G$ so that $e$ is in at least one
MST of the resulting graph, because we know that it is impossible to disconnect all
paths from $u$ to $v$ that are made up of edges that all have weight less than
$d$ by removing at most $k$ edges while still keeping the resulting graph
connected.

%%%%%%%%%%%%%%%%%%%%%%%%%%%%%% Problem 3
\section*{Problem 3}
Here is how we do this. We build a flow network $(V, E, s, t, c)$.
Here is how we structure it. For each producer we have a vertex
$v_{p_i}$ that represents the $i$th producer (so we have $m$ of these
vertices). The capacity of the edges from $s$ to each $v_{p_i}$ are of capacity
$0$. For each producer we also have a vertex $v_{p_i}'$. We connect
each $v_{p_i}$ to the respective $v_{p_i}'$ (so connect $v_{p_1}$ to $v_{p_1}'$,
$v_{p_2}$ to $v_{p_2}'$, etc.). The capacity of the edges that make
these connections be $\infty$. Then for each actor we have
a vertex $v_{a_i}$ that represents the $i$th actor
(so we have $n$ of these vertices). Then we connect
each vertex $v_{p_i}$ to every $v_{a_i}$ where the $i$th actor is one of the $i$th
producer's favorite actors. That is, we basically connect the producers to
their favorite actors. The capacity of the edges that make these connections
is $\infty$. Then, for each vertex $v_{p_i}'$, we connect it to vertex $t$.
The capacity of the edges that make these connections is $-X_i$. Then,
for each vertex $v_{a_i}$, we connect it to vertex $t$. The capacity
of the edges that make these connections is $Y_i$. Then our setup is done.

Now all we need to do is find the edges that cross the min cut of this graph.
This can be done in the following way. Find a cut whose size is equivalent to
the size of the maximum flow (a min cut). This can be done by using Ford
Fulkerson to find the max flow, then seeing how much flow goes out of the
source node to get the size of the max flow, then using a BFS to find a cut
whose size is equivalent to the size of the max flow. Then we can iterate over the
edges and find the edges that cross the cut. Call this set of edges $C$. Then
we can just iterate over the edges in $C$, find all vertices of the form
$v_{p_i}'$ contained in these edges, and add those $i$s to a list $S$.
Then Victor, Henry, and Adam
can just iterate through the indices contained in $S$ and choose those producers.
That is, if $S$ contains the indices $x_1, x_2, \cdots, x_k$, Victor, Henry, and
Adam should choose the $x_1$th, $x_2$th,..., $x_k$th producers.

Now we will briefly explain why this works. We basically have that representing
this problem as our graph gives us the following. If the actors that a producer
$p_i$ wants are too expensive, taking into account the fact that other producers
may want those actors as well, then we can just separate $s$ and $v_{p_i}$ in
our cut with zero cost. Else, we will include the producer since we will cut
along the edges with costs of $X_i$s and $Y_i$s (since the investment of the
producer is more than the cost of the actors in this case, the cost of this part
of the cut will be negative), which include $v_{p_i}'$.
We also have that our algorithm takes into account actor overlap; that is, if
multiple producers want the same actor, we only include that cost once.
So clearly, our algorithm functions as intended.

%%%%%%%%%%%%%%%%%%%%%%%%%%%%%% Problem 4
\section*{Problem 4}
\begin{description}
    \item[(a)] We will prove this is true by induction on the size of odd
        length cycles. Our base case will be cycles of length $5$. To
        prove this, consider an arbitrary cycle $(x_1x_2x_3x_4x_5)$ of
        length $5$. We have that this is just the same as the composition
        $(x_1x_2x_3)(x_3x_4x_5)$. So the base case is satisfied. Now assume
        that any odd cycle of length $k$ can be expressed as a composition of
        cycles of length $3$. Now we must show that odd cycles of length
        $k+2$ can be expressed as a composition of cycles of length $3$.
        That is, we want to show that cycles of the form $(x_1x_2\cdots x_{k+2})$
        can be expressed as a composition of cycles of length $3$. We have that
        $(x_1x_2\cdots x_{k+2})$ can be expressed as
        $(x_1x_2\cdots x_k)(x_kx_{k+1}x_{k+2})$. And due to our inductive
        assumption $(x_1x_2\cdots x_k)$ can be expressed as a combination of
        cycles of length $3$. So we can conclude that odd cycles of length
        $k+2$ can be expressed as a composition of cycles of length $3$, and
        our inductive proof is complete.
    \item[(b)]
        We have that every decomposition of an even permutation $\alpha$ into
        transpositions consists of an even number of elements. So all we need
        to do is to show that every pair of transpositions can be turned into
        a composition of cycles of length three. So, consider the arbitrary
        transposition pair $(x_1x_2)(x_3x_4)$. We have that
        \[ (x_1x_2)(x_3x_4) = (x_1x_4x_3)(x_1x_2x_3) \]
        We can see that this is true because the composition on the left
        effectively swaps $x_1$ with $x_2$ and $x_3$ with $x_4$. So now
        we can just apply this equality to every pair of transpositions in
        the decomposition of $\alpha$. So clearly, it is true that every even
        transposition can be expressed as a composition of cycles of length
        three.
\end{description}

%%%%%%%%%%%%%%%%%%%%%%%%%%%%%% Problem 5
\section*{Problem 5}
We want to prove that if a permutation $\pi \in S_n$ (for $n \geq 3$) satisfies
$\pi \tau = \tau \pi$ for every transposition $\tau \in S_n$, then
$\pi = \text{id}$. To prove this, we will prove the contrapositive; that is,
if $\pi \neq \text{id}$, then there exists a $\tau \in S_n$ for which
$\tau \pi \tau^{-1} \neq \pi$ for $\pi \in S_n$, $n \geq 3$.
So, consider an arbitrary $\pi \in S_n$ for
which $n \geq 3$, $\pi = (x_1x_2\cdots x_k)$. Then we will choose
$\tau = (x_1x_k) \implies \tau^{-1} \implies (x_kx_1)$. Then we have that
applying $\tau^{-1}$, then $\pi$, then $\tau$ to some arbitrary sequence of
numbers $\cdots x_1 \cdots x_k \cdots$ results in the following.
\[ \text{applying } \tau^{-1} \implies \cdots x_k \cdots x_1 \cdots \]
\[ \text{applying } \pi \implies \cdots x_1 \cdots x_2 \cdots \]
\[ \text{applying } \tau \implies \cdots x_k \cdots x_2 \cdots \]
But we have that just applying $\pi$ to that same sequence of numbers
results in the following.
\[ \text{applying just } \pi \implies \cdots x_2 \cdots x_1 \cdots \]
So clearly, we have that $\tau \pi \tau^{-1} \neq \pi$. Thus we have
proved the contrapositive and thus proved the original statement.

\end{document}
